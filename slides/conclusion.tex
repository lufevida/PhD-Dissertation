
%------------------------------------------------------------------------
\begin{frame}
	\frametitle{Main Objectives}
	\begin{itemize}
		\item Study the application of self-derivation matrices to the algorithmic composition of acoustic and electroacoustic music
		\item Generalize the work of Starr (1984) to arbitrary equal-temperament systems
		\item Provide a solid mathematical foundation that describes the derivation of multiple order-number function arrays (required for non-brute-force algorithms)
		\item Confront findings with those of Kowalski (1987)
	\end{itemize}
\end{frame}

%------------------------------------------------------------------------
\begin{frame}
	\frametitle{Specific Questions}
	\begin{itemize}
		\item For any given row, determine whether it is capable of producing any known form of self-derivation
		\item Conversely, given some self-derivation procedure, provide a complete description of the class of rows that participate therein
		\item Extend the above to cases where the derived row is not in the same row class as the originating row
		\item Given a pattern of order numbers, understand what rows, and under which operations, yield known forms of derivation and self-derivation
		\item Investigate how chains of pitch-class operations are induced by derivation procedures and conversely
	\end{itemize}
\end{frame}

%------------------------------------------------------------------------
\begin{frame}
	\frametitle{Specific Questions}
	\begin{itemize}
		\item Make precise what the constraints regarding the choice of operation are for folded and shifted derivations
		\item Investigate the need for commutativity, and determine how the aforementioned forms support derivation without the retrograde
		\item Describe the role of the cyclic rotation operator, particularly in what regards folded self-derivations
		\item Generalize mallalieu rows to arbitrary equal temperament systems, giving a precise count and description of their untransposed representatives
		\item Investigate generalizations of the mallalieu property by relaxing the requirement that we get a copy of the row at \emph{every} index
		\item Describe mallalieu rows arising from operations other than transposition
	\end{itemize}
\end{frame}

%------------------------------------------------------------------------
\begin{frame}
	\frametitle{Expected Results}
	\begin{itemize}
		\item Provide a complete classification of rows that support self-derivation in matrices of all sizes
		\item Classify the algebraic operations that afford self-derivation
		\item Create general-purpose and non-brute-force algorithms to implement findings in the programming languages that pertain to musicians the most
		\item Compose pieces that illustrating the real-life use of such algorithms
		\item Provide criticism as to where such compositional practices situate within 21\textsuperscript{st}-century music composition
	\end{itemize}
\end{frame}

%------------------------------------------------------------------------
\begin{frame}
	\frametitle{Chapter Outline}
	\begin{itemize}
		\item Introduction
		\item Literature Review
		\item Theoretical Framework
		\item Summary of Results
		\item Compositional Applications
	\end{itemize}
\end{frame}
