
%------------------------------------------------------------------------
\begin{frame}
	\frametitle{Shifted Derivation}
	\begin{table}
    	\caption{Schematics of a shifted derivation procedure involving the retrograde. The asterisk indicates there is no requirement we derive the same row class in both foldings, in which case $V$ would be replaced by some other row $V^*$ and $G$ could be the identity.}
    	\centering
    	\vspace{12pt}
    	\begin{tabular}{ c | c c c }
        	\hline%\noalign{\smallskip}
        	& $S$ & $\R \circ F(S)$ & \\
        	\hline
        	$V$ & $V_1$ & $V_2$ & \\
        	$\R \circ F(V)$ & $\R \circ F(V_2)$ & $\R \circ F(V_1)$ & \\
        	$\R \circ G(V)^*$ && $\R \circ G(V_2)^*$ & $\R \circ G(V_1)^*$ \\
        	$GF(V)^*$ && $GF(V_1)^*$ & $GF(V_2)^*$ \\
        	\hline
        	&& $GF(S)$ & $\R \circ G(S)$ \\
        	\hline
    	\end{tabular}
	\end{table}
\end{frame}

%------------------------------------------------------------------------
\begin{frame}
	\frametitle{Shifted Derivation}
	\begin{block}{Example}
		Consider the row $S = \{ 0, 1, 7, 2, 10, 9, 11, 4, 8, 5, 3, 6 \}$ and the combination matrix given by $\R\T_{10}\I(S)$ against $\T_{11}(S)$:
    	\begin{equation*}
        	\left[
        	\begin{array}{cccccccc|cccccccc}
            	4 && 7 && 5 && 2 && 6 & 11 & 1 & 0 & 8 & 3 & 9 & 10 \\
            	11 & 0 & 6 & 1 & 9 & 8 & 10 & 3 && 7 && 4 && 3 && 5
        	\end{array}
        	\right]
    	\end{equation*}
	\end{block}
\end{frame}

%------------------------------------------------------------------------
\begin{frame}
	\frametitle{Shifted Derivation}
	\begin{block}{Example}
		Deriving the row $V = V_1 | V_2 = \{ 1, 7, 2, 9, 8, 3 \} | \{ 4, 5, 6, 11, 0, 10 \}$ from $S | \R\T_{10}\I(S)$ yields the following shifted derivation:
		\begin{equation*}
    	\begin{adjustbox}{width=\textwidth}
        	$\left[
        	\begin{array}{cccccccccccc|cccccccc|c}
            	& 1 & 7 & 2 && 9 &&& 8 && 3 && 4 &&&& 5 &&&& 6 \\
            	0 &&&& 10 && 11 & 4 && 5 && 6 & && 7 &&&& 2 && \\
            	\hline
            	&&&&&&&&&&&& & 0 & 6 & 1 && 8 &&& 7 \\
            	&&&&&&&&&&&& 11 &&&& 9 && 10 & 3 &
        	\end{array}
        	\right. \cdots$
    	\end{adjustbox}
    	\end{equation*}
    	\begin{equation*}
    	\begin{adjustbox}{width=\textwidth}
        	$\cdots \left. \begin{array}{c|cccccccc|cccccccccccc}
            	& 6 & 11 && 0 &&&& 10 &&&&&&&&&&& \\
            	& && 1 && 8 & 3 & 9 & &&&&&&&&&&& \\
            	\hline
            	& 7 &&&& 3 &&& & 3 && 4 && 5 & 10 && 11 &&&& 9 \\
            	3 &&& 4 &&&& 5 & && 6 && 1 &&& 0 && 7 & 2 & 8
        	\end{array} \right]$
    	\end{adjustbox}
    	\end{equation*}
	\end{block}
\end{frame}

%------------------------------------------------------------------------
\begin{frame}
	\frametitle{Self Derivation}
	\begin{table}
    	\caption{Schematics of a self-derivation procedure involving the retrograde.}
    	\centering
    	\vspace{12pt}
    	\begin{tabular}{c|cc}
        	\hline
        	& $G(S)$ & $\R \circ FG(S)$ \\
        	\hline
        	$S$ & $S_1$ & $S_2$ \\
        	$\R \circ F(S)$ & $\R \circ F(S_2)$ & $\R \circ F(S_1)$ \\
        	\hline
    	\end{tabular}
	\end{table}
\end{frame}

%------------------------------------------------------------------------
\begin{frame}
	\frametitle{Self Derivation}
	\begin{block}{Example}
		Let $S = S_1 | S_2 = \{ 3, 8, 1, 0, 9, 6 \} | \{ 4, 7, 10, 5, 2, 11 \}$. In particular, both $S_1$ and $S_2$ are invariant under $\T_9\I$. What is not obvious is that we can derive $S$ and $\T_9\I(S)$ from a combination matrix where the first column is $\T_7(S)$ and the second is $\T_9\I \circ \R\T_7(S) = \R\T_2\I(S)$:
    	\begin{equation*}
    	\begin{adjustbox}{width=\textwidth}
        	$\left[
        	\begin{array}{cccccccccccc|cccccccccccc}
            	& 3 & 8 &&& 1 &&&& 0 & 9 & 6 &&&& 4 & 7 & 10 && 5 & 2 &&& 11 \\
            	10 &&& 7 & 4 && 11 & 2 & 5 &&&& 3 & 0 & 9 &&&& 8 &&& 1 & 6 &
        	\end{array}
        	\right]$
    	\end{adjustbox}
    	\end{equation*}
		Here we have $F = \T_9\I$ and $G = \T_7$. It is not the case that $F$ and $G$ commute, as $\T_2\I = F \circ G \ne G \circ F = \T_4\I$.
	\end{block}
\end{frame}

%------------------------------------------------------------------------
\begin{frame}
	\frametitle{Self Derivation}
	\begin{block}{Example}
		Let $S = \{ 0, 11, 5, 10, 4, 2, 7, 9, 8, 3, 6, 1 \}$ and consider the following combination matrix given by $T_2(S) | \R\T_2(S)$, whose derived rows are $S$ and $\R(S)$:
    	\begin{equation*}
    	\begin{adjustbox}{width=\textwidth}
        	$\left[
        	\begin{array}{cccccccccccc|cccccccccccc}
            	& 0 && 11 & 5 &&& 10 && 4 && 2 && 7 && 9 && 8 & 3 &&& 6 && 1 \\
            	1 && 6 &&& 3 & 8 && 9 && 7 && 2 && 4 && 10 &&& 5 & 11 && 0 &
        	\end{array}
        	\right]$
    	\end{adjustbox}
    	\end{equation*}
		Subjecting the entire matrix to $T_1$ yields:
    	\begin{equation*}
    	\begin{adjustbox}{width=\textwidth}
        	$\left[
        	\begin{array}{cccccccccccc|cccccccccccc}
            	& 1 && 0 & 6 &&& 11 && 5 && 3 && 8 && 10 && 9 & 4 &&& 7 && 2 \\
            	2 && 7 &&& 4 & 9 && 10 && 8 && 3 && 5 && 11 &&& 6 & 0 && 1 &
        	\end{array}
        	\right]$
    	\end{adjustbox}
    	\end{equation*}
	\end{block}
\end{frame}

%------------------------------------------------------------------------
\begin{frame}
	\frametitle{Self Derivation}
	\begin{block}{Example}
		We can then pull an entire matrix from the first row of above:
    	\begin{equation*}
    	\begin{adjustbox}{width=\textwidth}
        	$\left[
        	\begin{array}{cccccccccccc|cccccccccccc|c}
            	&&& 0 &&&& 11 && 5 &&&&&& 10 &&& 4 &&&&& 2 & \\
            	& 1 &&& 6 &&&&&&& 3 && 8 &&&& 9 &&&& 7 &&& 2 \\
            	2 && 7 &&& 4 & 9 && 10 && 8 && 3 && 5 && 11 &&& 6 & 0 && 1 &&
        	\end{array}
        	\cdots \right.$
    	\end{adjustbox}
    	\end{equation*}
	\end{block}
\end{frame}
