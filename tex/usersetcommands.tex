
% optional command to prevent a word from breaking across a line
\hyphenchar\font=-1

% Commands to produce proper bullet list
\newlength{\widthOfItem}
\let\Itemize=\itemize
\let\endItemize=\enditemize
\renewenvironment{itemize}{%
	\begin{Itemize}
		\setlength{\itemsep}{0.5\baselineskip}
		\setlength{\labelwidth}{2em}
		\setlength{\listparindent}{.32in}%
		\setlength{\leftmargin}{.32in}
		\setlength{\rightmargin}{0in}
		\settowidth{\widthOfItem}{\labelitemi}
		\setlength{\labelsep}{\leftmargin-\widthOfItem}
		\renewcommand{\labelitemii}{--}
		\singlespacing}{%
	\end{Itemize}
}

% shortcut for setting up inserting \prime command in mathmode to avoid errors
\newcommand{\p}{^{\prime}}

% shortcuts for prime color text
\newcommand{\red}{\textcolor[rgb]{1.00,0.00,0.00}}
\newcommand{\green}{\textcolor[rgb]{0.00,1.00,0.00}}
\newcommand{\blue}{\textcolor[rgb]{0.00,0.00,1.00}}

% Shorcut commands for mathmatical formulas
\newcommand{\latex}{\LaTeX 2\ensuremath{\epsilon}}

\DeclareMathOperator{\T}{T}
\DeclareMathOperator{\I}{I}
\DeclareMathOperator{\M}{M}
\DeclareMathOperator{\R}{R}
\DeclareMathOperator{\Aut}{Aut}
\DeclareMathOperator{\Fix}{Fix}

% To link the theorem to each chapter uncomment the chapter option
\newtheorem{theorem}{Theorem}[section]
\newtheorem{lemma}[theorem]{Lemma}
\newtheorem{corollary}[theorem]{Corollary}
\newtheorem{definition}[theorem]{Definition}
\newtheorem{conjecture}[theorem]{Conjecture}
\newtheorem{proposition}[theorem]{Proposition}
\newtheorem{algorithm}[theorem]{Algorithm}
\newtheorem{example}[theorem]{Example}
