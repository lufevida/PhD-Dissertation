%--------------------------------------------------------------------------
\chapter{LITERATURE REVIEW}

%--------------------------------------------------------------------------
\section{Discovering Derivation}

The beginnings of derivation techniques can be traced back to Donald Martino's \emph{The Source Set and Its Aggregate Formations} \cite{Martino1961}, a paper whose main purpose is another, namely to generalize the construction of columnar realizations, although this latter term appears to have been coined by \cite{Starr1984}. The techniques employed by Martino are heavily influenced by Babbitt's ideas on combinatoriality \cite[224]{Martino1961}, and make extensive use of tables. Below is a fragment of the source hexachords table given in \cite[229]{Martino1961}. The author, in the interest of generalizing the procedure, provides also trichordal, tetrachordal, and even pentachordal combinatoriality tables, as well as a somewhat brief discussion on uneven partitions of a row and their combinatoriality implications \cite[267]{Martino1961}.

\begin{table}[htbp]
    \caption[Martino's Source Hexachords]{Fragment of a table for consulting source hexachords in \cite[229]{Martino1961}.}
    \centering
    \vspace{12pt}
    \begin{tabular}{c|cccccc}
        \hline\\
        No. & Set & Interval Vector & TTO's \\\\
        \hline\\
        A1 & $\{0,1,2,3,4,5\}$ & $\langle 5,4,3,2,1,0 \rangle$ & $\{\T_6,\T_{11}\I,\R\T_{11},\R\T_6\I\}$ \\\\
        B2 & $\{0,2,3,4,5,7\}$ & $\langle 3,4,3,2,3,0 \rangle$ & $\{\T_6,\T_{1}\I,\R\T_{1},\R\T_6\I\}$ \\\\
        3 & $\{0,1,3,4,5,8\}$ & $\langle 3,2,3,4,3,0 \rangle$ & $\{\T_6,\R\T_2\}$ \\\\
        E4 & $\{0,1,4,5,8,9\}$ & $\langle 3,0,3,6,3,0 \rangle$ & $\{\T_{2,6,10},\T_{3,7,11}\I,\R\T_{3,7,11},\R\T_{2,6,10}\I\}$ \\\\
        C5 & $\{0,2,4,5,7,9\}$ & $\langle 1,4,3,2,5,0 \rangle$ & $\{\T_6,\T_{3}\I,\R\T_{3},\R\T_6\I\}$ \\\\
        \hline
    \end{tabular}
\end{table}

Martino acknowledges the aggregates formed by combinatoriality are not necessarily ordered \cite[228]{Martino1961}, rather seeing it from the bright side of diversity, in which the set union of a columnar aggregate being capable of \emph{deriving} a different row class is actually more appealing than self-similarity, which is not at all considered. The same passage brings, to our knowledge, the first \emph{explicit} mention in the literature of a technique for deriving new rows from columnar aggregates in more generality than what we see in Schoenberg's fourth string quartet, as seen in \ref{ex:martino-derivation}. In mentioning derivation, Martino juxtaposes it to another, in his view, way of progressing through hexachords, namely the \emph{fragmentation} or partitioning of the original series, ultimately deeming both procedures essentially the same, as derivation via aggregate realizations can surely be seen as the fragmentation of the (new) series obtained vertically.

\begin{example}
	\label{ex:martino-derivation}
	\cite[230]{Martino1961}
	Let $S = \{ 0, 4, 11, 3, 1, 2, 5, 6, 9, 8, 10, 7 \}$ and consider the trivial  combination matrix given below.
	\begin{equation}
    	\hat{A} = [\hat{A}_1 | \hat{A}_2 | \hat{A}_3 | \hat{A}_4] = \left[
    	\begin{array}{ccc|ccc|ccc|ccc}
        	0 & 4 & 11 & 3 & 1 & 2 & 5 & 6 & 9 & 8 & 10 & 7 \\
        	7 & 10 & 8 & 9 & 6 & 5 & 2 & 1 & 3 & 11 & 4 & 0
    	\end{array}
    	\right] \enspace.
	\end{equation}
	In particular, the hexachords given by $\hat{A}_i$ can be combined with their transforms under $\T_1\I$, yielding the following matrix:
	\begin{equation}
    	A = [A_1 | A_2 | A_3 | A_4] = \left[
    	\begin{array}{ccc|ccc|ccc|ccc}
        	0 & 4 & 11 & 3 & 1 & 2 & 5 & 6 & 9 & 8 & 10 & 7 \\
        	7 & 10 & 8 & 9 & 6 & 5 & 2 & 1 & 3 & 11 & 4 & 0 \\
        	\hline
        	1 & 9 & 2 & 10 & 0 & 11 & 8 & 7 & 4 & 5 & 3 & 6 \\
        	6 & 3 & 5 & 4 & 7 & 8 & 11 & 0 & 10 & 2 & 9 & 1
    	\end{array}
    	\right] \enspace.
	\end{equation}
	We can then \textbf{almost} derive transforms of the row $Q = \{ 7, 0, 10, 6, 4, 1, 3, 5, 8, 9, 2, 11 \}$ from the columns of $A$, except for a single symmetry between 6 and 5, as seen below.
	\begin{equation}
        \left[
        \begin{array}{cccccccccccc|cccccccccccc}
            & 0 &&& 4 &&&&&&& 11 &&&& 3 & 1 &&& 2 &&&& \\
            7 && 10 &&&&&& 8 &&& && 9 &&&&&&& \boxed{6} & \boxed{5} && \\
            \hline
            &&&&& 1 &&&& 9 & 2 & &&&&&& 10 & 0 &&&& 11 & \\
            &&& 6 &&& 3 & 5 &&&& & 4 && 7 &&&&&&&&& 8
        \end{array}
        \right. \cdots
    \end{equation}
    This apparent failure did not seem to bother Martino at all, as his focus remained in establishing a concrete foundation for combinatoriality. In order to extend the above procedure to eight rows of counterpoint, some adjustments must be made, namely we need columnar aggregates whose rows are allowed to have different numbers of elements, as seen below. This second folding is otherwise accomplished much the same way, by choosing a hexachord from $A_1$, and determining its combinatoriality properties via table lookup.
    \begin{equation}
    	\left[
    	\begin{array}{cc|cc|cc|cc|}
        	0 & 4 & 11 && 3 & 1 & 2 & \\
        	7 & 10 & 8 && 9 && 6 & 5 \\
        	1 && 9 & 2 & 10 && 0 & 11 \\
        	6 && 3 & 5 & 4 & 7 & 8 & \\
        	2 && 6 & 1 & 5 && 3 & 4 \\
        	9 && 0 & 10 & 11 & 8 & 7 & \\
        	3 & 11 & 4 && 0 & 2 & 1 & \\
        	8 & 5 & 7 && 6 && 9 & 10
    	\end{array}
    	\right. \cdots
	\end{equation}
\end{example}

It is easy to understand why Martino was so motivated by creating as much diversity as possible from a single row in a structured manner. If not because creating variation upon some fixed foundation has been a constant in music composition since Bach, serialism up to that point had seen countless pieces in which the very same series was presented over and over again, frequently in the same $\R\T_n\I$ form for entire passages. As innovative as it is, not even the first movement of Schoenberg's fourth string quartet escapes this paradigm. The idea of having a systematic approach to syntactically moving from one row class to another, although latent in late Schoenberg, ultimately defines one one of the most remarkable trends in the next chapter of serialism. Martino actually manages to solve both problems at once with derivation, as it gives the composer the opportunity to focus on the derived rows, while using a somewhat more abstract row to generate syntax, even though we only see this potential explored in his later pieces, and not quite yet in \cite{Martino1961}. For now, what is most important is to derive as much harmonic diversity from the rigidity of an omnipresent series as possible.

What \cite{Starr1984} terms \emph{skewed polyphonization}, and what we called shifted derivation in Sec.~\ref{shifted-derivation}, has too an origin in what Martino calls, in turn, \emph{oblique combination}. Ex.~\ref{ex:oblique} deals with the tetrachordal case, although other cases, including those where the overlap occurs at unequal partitionings of the row are considered. Similarly to our intuition in Sec.~\ref{shifted-derivation}, oblique combinations are seen as special, yet straightforward cases of other types of combinatoriality \cite[267]{Martino1961}.

\begin{example}
	\cite[241]{Martino1961}
	\label{ex:oblique}
	Let $S = S_1 | S_2 | S_3$ be a twelve-tone row, partitioned into tetrachords. Let $f$ and $g$ be TTO's, and considered the following oblique combination matrix:
	\begin{equation}
    	\left[
    	\begin{array}{c|c|c|c|c}
        	S_1 & S_2 & S_3 & . & . \\
        	. & f(S_3) & f(S_2) & f(S_1) & . \\
        	. & . & g(S_1) & g(S_2) & g(S_3)
    	\end{array}
    	\right] \enspace.
	\end{equation}
	It follows:
	\begin{enumerate}[i.]
		\item If $S_1 = \T_i\I(S_1)$ \textbf{only} for some $i$, then $f = \R\T_j$ for some $j$, and $g = \T_k\I$ for some $k$;
		\item If $S_3 = \T_i\I(S_3)$ \textbf{only} for some $i$, then $f = \R\T_j\I$ for some $j$, and $g = \T_k$ for some $k$;
		\item If $S_1 = \T_i\I(S_1)$ \textbf{only} for some $i$ and $S_3 = \T_j\I(S_3)$ for some $j$, then $f = \R\T_k\I$ for some $k$, and $g = \T_\ell\I$ for some $\ell$;
		\item If $S_1 = \T_i\I(S_1)$ \textbf{only} for some $i$, $S_3 = \T_j\I(S_3)$ for some $j$, and $S_1 = \T_k(S_3)$ for some $k$, then $f = \T_\ell\I$ or $f = \R\T_\ell\I$ for some $\ell$, and $g = \T_m\I$ or $g = \R\T_m\I$ for some $m$;
		\item If $S_1 = \T_i(S_3)$ \textbf{only} for some $i$, then $f = \T_j$ or $f = \R\T_j$ for some $j$, and $g = \T_k$ or $g = \R\T_k$ for some $k$;
		\item If $S_3 = \T_i\I(S_1)$ \textbf{only} for some $i$, then $f = \T_j\I$ or $f = \R\T_j$ for some $j$, and $g = \T_k$ or $g = \R\T_k\I$ for some $k$.
	\end{enumerate}
\end{example}
