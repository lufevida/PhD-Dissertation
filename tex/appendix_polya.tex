
%--------------------------------------------------------------------------
\chapter{Polya's Enumeration Formula}

%--------------------------------------------------------------------------
\begin{definition}
	\cite[85]{Aigner2007}
	Let $N$ be a set of $n$ beads and $R$ be a set of $r$ colors. A colored necklace is
	a function $f : N \to R$. Denote the set of all such functions by $R^N$, so that
	$|R^N| = r^n$. Define the \textbf{weight} associated to a coloring $f$ by
	\begin{equation}
		w(f) = \prod_{i \in N} x_{f(i)} \enspace,
	\end{equation}
	where $x_j$ is a variable associated with the color $j \in R$.
\end{definition}

%--------------------------------------------------------------------------
\begin{proposition}
	\label{coloring}
	\cite[110]{Rotman1967}
	Let $G$ be a group and $X = \{ 1, \cdots, n \}$ be a set. Let $\mathcal{C}$ be a set
	of $q$ colors. Then $G$ acts on the set $\mathcal{C}^n$ of $n$-tuples of colors by
	\begin{equation}
		\tau(c_1, \cdots, c_n) = (c_{\tau 1}, \cdots, c_{\tau n}),
		\forall \tau \in G \enspace.
	\end{equation}
\end{proposition}

%--------------------------------------------------------------------------
\begin{proposition}
	\cite[85]{Aigner2007}
	\cite[110]{Rotman1967}
	If $G$ is a group acting on the set $N$, then two colorings $f$ and $f^\prime$
	are equivalent whenever $f = f^\prime \circ g$ for some $g \in G$. This is an
	equivalence relation that partitions $R^N$ into equivalence classes denoted
	\textbf{patterns}. Under the conditions of \ref{coloring}, an orbit
	$(c_1, \cdots, c_n) \in \mathcal{C}^n$ is called a $(q, G)$-\textbf{coloring} of $X$.
	Moreover, two equivalent colorings have the same weight, so that we may refer to
	the weight of a class of colorings rather than the weights of its representatives.
\end{proposition}

%--------------------------------------------------------------------------
\begin{definition}
	\cite[85]{Aigner2007}
	Let $N$ be a set of $n$ beads and $R$ be a set of $r$ colors. Let $G$ be a group acting
	on the set $N$ and $x_j$ the variable associated with the color $j \in R$. Let
	$\mathcal{M}$ be the set of of patterns. Define the pattern \textbf{enumerator} by
	\begin{equation}
		w(R^N, G) = \sum_{M \in \mathcal{M}} w(M) \enspace.
	\end{equation}
	In particular, when $x_j = 1$ for all $j \in R$, we get $w(R^N, G) = |\mathcal{M}|$.
\end{definition}

%--------------------------------------------------------------------------
\begin{definition}
	\cite[86]{Aigner2007}
	Let $G$ be a group acting on a set $X$. Define for every $g \in G$ its
	\textbf{fixed-point set} by 
	\begin{equation}
		\Fix(g) = \{ x \in X : gx = x \} \enspace.
	\end{equation}
\end{definition}

%--------------------------------------------------------------------------
\begin{lemma}
	\cite[112]{Rotman1967}
	Let $G < S_n$ be a group and let $\mathcal{C}$ be a set of $q$ colors. For $\tau \in G$,
	\begin{equation}
		|\Fix(\tau)| = q^{t(\tau)} \enspace,
	\end{equation}
	where $t(\tau)$ is the number of cycles in the complete factorization of $\tau$.
\end{lemma}

%--------------------------------------------------------------------------
\begin{lemma}[Burnside]
	\cite[109]{Rotman1967}
	\cite[251]{Tucker1974}
	If $G$ acts on a finite set $X$, then the number of orbits $N$ is
	\begin{equation}
		N = \frac{1}{|G|} \sum_{\tau \in G} |\Fix(\tau)| \enspace.
	\end{equation}
\end{lemma}

%--------------------------------------------------------------------------
\begin{corollary}
	\cite[112]{Rotman1967}
	Let $G$ be a group acting on a finite set $X$. The number $N$ of $(q, G)$-colorings
	of $X$ is
	\begin{equation}
		N = \frac{1}{|G|} \sum_{\tau \in G} q^{t(\tau)} \enspace.
	\end{equation}
\end{corollary}

%--------------------------------------------------------------------------
\begin{corollary}
	\cite[54]{Reiner1985}
	\cite[127]{FripertingerLackner2015}
	The number of $n$-tone rows under $\R \T_n \I$ is
	\begin{equation}
		\begin{cases}
			\frac{1}{4} \left[ (n - 1)! + (n - 1) (n - 3) \cdots (2) \right]
			& n \text{ odd} \\
			\frac{1}{4} \left[ (n - 1)! + (n - 2) (n - 4) \cdots (2) (1 + \frac{n}{2})
			\right] & n \text{ even} \\
		\end{cases}
	\end{equation}
	In particular, there are 9985920 twelve-tone rows.
\end{corollary}

%--------------------------------------------------------------------------
\begin{definition}
	\cite[87]{Aigner2007}
	Let $g$ be a group acting on $R^N$. Define the \textbf{cycle indicator} of $G$ by
	\begin{equation}
		P_G(z_1, z_2, \cdots, z_n) = \frac{1}{|G|} \sum_{g \in G} z_1^{b_1(g)} z_2^{b_2(g)}
		\cdots z_n^{b_n(g)} \enspace,
	\end{equation}
	where $z_i^{b_i(g)}$ corresponds to the number of cycles in the complete factorization
	of $g$ that have length $i$.
\end{definition}

%--------------------------------------------------------------------------
\begin{example}
	The cycle indicator of $S_3$ is
	\begin{equation}
		P_{S_3}(x_1, x_2, x_3) = \frac{1}{6}(x_1^3 + 3 x_1^1 x_2^1 + 2 x_3^1) \enspace,
	\end{equation}
	since there are two elements in $S_3$ that comprise one cycle of length three,
	three elements that comprise one cycle of length one, and one cycle of length two,
	and one element that comprises three cycles of length one.
\end{example}

%--------------------------------------------------------------------------
\begin{example}
	Let $C_n$ be the cyclic group of order $n$. Then
	\begin{equation}
		P_{C_n} = \frac{1}{n} \sum_{d | n} \varphi(d) z_d^{n / d} \enspace.
	\end{equation}
\end{example}

%--------------------------------------------------------------------------
\begin{example}
	For $S_n$ we have
	\begin{equation}
		P_{S_n} = \sum_{j_1 + 2j_2 + \cdots + nj_n = n}
		\frac{1}{\prod_{k = 1}^n k^{j_k} j_k!} \prod_{k = 1}^n a_k^{j_k} \enspace.
	\end{equation}
	In words, there is a summand for each conjugacy class in $S_n$, and we divide each
	summand by the size of its corresponding conjugacy class.
\end{example}

%--------------------------------------------------------------------------
\begin{example}
	For $D_n$ we have
	\begin{equation}
		P_{D_n} = \frac{1}{2} P_{C_n} +
		\begin{cases}
			\frac{1}{2} a_1 a_2^{(n - 1) / 2} & n \text{ odd} \\
			\frac{1}{4} (a_1^2 a_2^{(n - 2) / 2} + a_2^{n / 2)} & n \text{ even}
		\end{cases}
	\end{equation}
	In particular, we obtain
	\begin{equation}
		P_{D_{24}}(1 + x, \cdots, 1 + x^{12}) = \frac{1}{24} ( x_1^{12} + 6 x_1^2 x_2^5 +
		7 x_2^6 + 2 x_3^4 + 2 x_4^3 + 2 x_6^2 + 4 x_{12} ) \enspace.
	\end{equation}
\end{example}

%--------------------------------------------------------------------------
\begin{example}
	%\cite[1]{WeiXu1993}
	\cite[120]{FripertingerLackner2015}
	For $\Aff_1(C_{12})$ we have
	\begin{multline}
		P_{\Aff_1(C_{12})}(1 + x, \cdots, 1 + x^{12}) = \frac{1}{48} (
		x_1^{12} + 2 x_1^6 x_2^3 + 3 x_1^4 x_2^4 + 6 x_1^2 x_2^5 \\ + 12 x_2^6 +
		4 x_2^3 x_6 + 2 x_3^4 + 8 x_4^3 + 6 x_6^2 + 4 x_{12} ) \enspace.
	\end{multline}
\end{example}

%--------------------------------------------------------------------------
\begin{theorem}[Polya]
	\cite[88]{Aigner2007}
	\cite[256]{Tucker1974}
	Let $N$ be a set of cardinality $n$ and $R$ be a set of cardinality $r$. Let $G$ be
	a group acting on the set $N$ and $x_j$ be arbitrary variables with $j \in R$.
	Let $w(R^N, G)$ be the pattern enumerator for the action of $G$ on $R^N$. Then
	\begin{equation}
		w(R^N, G) = \sum_{M \in \mathcal{M}} w(M) = P_G(\sum_{j \in R} x_j,
		\sum_{j \in R} x_j^2, \cdots, \sum_{j \in R} x_j^n) \enspace.
	\end{equation}
\end{theorem}

%--------------------------------------------------------------------------
\begin{corollary}
	\cite[89]{Aigner2007}
	\cite[254]{Tucker1974}
	Let $x_j = 1$ for all $j \in R$. Then $\sum_{j \in R} x_j^k = |R| = r$ for all $k$, hence
	\begin{equation}
		P_G(r, \cdots, r) = |\mathcal{M}| \enspace.
	\end{equation}
\end{corollary}

%--------------------------------------------------------------------------
\begin{corollary}
	\label{oneColor}
	\cite[89]{Aigner2007}
	Let $|R| = r = 2$. Let $x_1 = x_{\text{white}} = x$ and
	$x_2 = x_{\text{black}} = 1$. Then
	\begin{equation}
		P_G(x + 1, x^2 + 1, \cdots, x^n + 1) = \sum_{k = 0}^n a_k x^k \enspace,
	\end{equation}
	where $a_k$ is the number of patterns in which the color white occurs exactly $k$ times.
\end{corollary}

%--------------------------------------------------------------------------
\begin{example}
	\cite[89]{Aigner2007}
	The number of necklaces with $n$ beads and $r$ colors is
	\begin{equation}
		\frac{1}{n} \sum_{d | n} \varphi(d) r_d^{n / d} \enspace.
	\end{equation}
\end{example}

%--------------------------------------------------------------------------
\begin{example}
	\cite[86]{Aigner2007}
	The definition of a weight function is useful in order to count how many necklaces
	contain precisely $x_j$ beads of color $j$. Set $W$ to be the color white and $B$
	to be the color black. Then a 4-bead necklace with exactly 2 white and 2 black beads
	is expressed by $W^2B^2$. If we do not wish to account for a particular color,
	per \ref{oneColor} we may assign its weight to 1. If we chose not to account for the color
	black, say, then the representation above would become simply $W^21^2 = W^2$.
\end{example}

%--------------------------------------------------------------------------
\begin{example}
	We use Polya's theorem to describe the pattern inventory of 3-bead necklaces under
	the action of $S_3$. Let the colors be $A$ and $B$. Then
	%\begin{equation}
		\begin{align}
			P_{S_3}(A + B, A^2 + B^2, A^3 + B^3)
			& = \frac{1}{6}[(A + B)^3 + 3 (A + B) (A^2 + B^2) + 2 (A^3 + B^3)] \\
			& = A^3 + A^2 B + A B^2 + B^3 \enspace.
		\end{align}
	%\end{equation}
	In words, we have one necklace with three $A$ beads, one with two $A$ and one $B$ bead,
	one with one $A$ and two $B$ beads, and one with three $B$ beads.
\end{example}

%--------------------------------------------------------------------------
\begin{example}
	We can use Polya's theorem to count chords in 12 tones that are equivalent under
	transposition by setting our set of colors to be $R = \{ r_0, r_1 \}$. We then disregard
	one of the colors, say $r_0$, by setting $w(r_0) = 1$ and $w(r_1) = C$. The group of
	transpositions is just $C_{12} = \mathbb{Z} / 12 \mathbb{Z}$, so we get
	%\begin{equation}
		\begin{align}
			P_{C_{12}}(1 + x, \cdots, 1 + x^{12})
			& = \frac{1}{12} \sum_{d \in \{ 1, 2, 3, 4, 6, 12 \}} \varphi(d) x_d^{12 / d} \\
			& = \frac{1}{12} \left( x_1^{12} + x_2^{6} + 2 x_3^{4} + 2 x_4^{3} +
			2 x_6^{2} + 4 x_{12} \right) \\
			& = \frac{1}{12} \left[ (1 + C)^{12} + (1 + C^2)^{6} + 2 (1 + C^3)^{4} \right. \\
			& \qquad \qquad
			+ \left. 2 (1 + C^4)^{3} + 2 (1 + C^6)^{2} + 4 (1 + C^{12}) \right] \\
			& = C^{12} + C^{11} + \cdots \\
			& \qquad \qquad
			\cdots + 80 C^6 + 66 C^5 + 43 C^4 + 19 C^3 + 6 C^2 + C^1 + C^0 \enspace.
		\end{align}
	%\end{equation}
	In particular, there are 43 tetrachords and 19 trichords that are transpositionally
	equivalent.
\end{example}

%--------------------------------------------------------------------------
\begin{example}
	\cite[53]{Reiner1985}
	More generally, the number of $k$-chords under the action of $C_n$ is
	\begin{equation}
		\frac{1}{n} \sum_{j | (n, k)} \varphi(j) {{n / j}\choose{k / j}} \enspace.
	\end{equation}
	Under the action of the dihedral group $D_{2n}$, we obtain
	\begin{equation}
		\begin{cases}
 			\frac{1}{2n} \left[ \sum_{j | (n, k)} \varphi(j) {{n / j}\choose{k / j}} +
 			n {{(n - 1) / 2} \choose{k / 2}} \right] & n \text{ odd} \\
 			\frac{1}{2n} \left[ \sum_{j | (n, k)} \varphi(j) {{n / j}\choose{k / j}} +
 			n {{n / 2} \choose{k / 2}} \right] & n \text{ even and } k \text{ even} \\
 			\frac{1}{2n} \left[ \sum_{j | (n, k)} \varphi(j) {{n / j}\choose{k / j}} +
 			n {{(n / 2) - 1} \choose{k / 2}} \right] & n \text{ even and } k \text{ odd}
 		\end{cases}
	\end{equation}
	In particular, the pattern inventory of $k$-chords in 12 tones that are equivalent under
	$\T_n \I$ is
	\begin{equation}
		P_{D_{24}}(1 + x, \cdots, 1 + x^{12}) = C^{12} + \cdots + 50 C^6 +
		38 C^5 + 29 C^4 + 12 C^3 + 6 C^2 + C^1 + C^0 \enspace.
	\end{equation}
\end{example}

%--------------------------------------------------------------------------
\begin{example}
	\cite[249]{Tucker1974}
	Consider the action of the dihedral group $D_8$ on the 16-element set of colorings in
	black and white of the corners of a square. The cycle indicator of $D_8$ is
	\begin{equation}
		P_{D_8}(x_1, x_2, x_3, x_4) =
		\frac{1}{8}(x_1^4 + 2x_1^2x_2 + 3x_2^2 + x_4^1) \enspace.
	\end{equation}
	Note that, in particular, we get no $x_3$ factors by Lagrange. The identity element
	in $D_8$ fixes all collorings, and is represented as a permutation by four cycles
	of length one. By Polya's formula, we make the substitution
	\begin{equation}
		x_1^4 = (B^1 + W^1)^4 = B^4 + 4 B^3 W + 6 B^2 W^2 + 4 B W^3 + W^4 \enspace,
	\end{equation}
	where the exponent outside the parenthesis is the number of cycles, and the exponents
	inside the parenthesis correspond to the lengths of the cycles, for each color.
	The element of order four in $D_8$ has one cycle of length four, so we get
	\begin{equation}
		x_4^1 = (B^4 + W^4)^1 = B^4 + W^4 \enspace,
	\end{equation}
	that is, the permutation $r = (1 \; 2 \; 3 \; 4)$ fixes the squares whose corners are
	all black or all white. Next, there are three elements in $D_8$ which comprise
	two cycles of length two, so each of those yield
	\begin{equation}
		x_2^2 = (B^2 + W^2)^2 = B^4 + 2 B^2 W^2 + W^4 \enspace.
	\end{equation}
	The last two elements in $D_8$ both have two cycles of length one, plus one cycle
	of length two, which gives
	\begin{equation}
		x_1^2x_2 = (B + W)^2 (B^2 + W^2) =
		B^4 + 2 B^3 W + 2 B^2 W^2 + 2 B W^2 + W^4 \enspace.
	\end{equation}
	Putting it all together, we get the following pattern inventory of orbits:
	\begin{equation}
		P_{D_8}(B + W, B^2 + W^2, B^3 + W^3, B^4 + W^4) =
		B^4 + 4 B^3 W + 6 B^2 W^2 + 4 B W^3 + W^4 \enspace.
	\end{equation}
\end{example}
