%--------------------------------------------------------------------------
\chapter{}

%--------------------------------------------------------------------------
\section{ELEMENTARY TWELVE-TONE THEORY}

% TODO Pitch, contour, and pitch class space
% TODO Octave equivalence
% TODO Set classes
% TODO Interval classes are equivalence relations
% TODO Twelve-tone rows
% TODO Twelve-tone operations on pitch classes and order numbers

%--------------------------------------------------------------------------
\begin{definition}
	Let $x$ and $y$ be arbitrary pitches. The \textbf{ordered pitch interval} between
	$x$ and $y$ is given by
	\begin{equation}
		i(x, y) = y - x \enspace.
	\end{equation}
	The \textbf{ordered pitch-class interval} between $x$ and $y$ is given by
	\begin{equation}
		i(\bar{x}, \bar{y}) = \bar{y} - \bar{x} \enspace.
	\end{equation}
	The \textbf{unordered pitch interval} between $x$ and $y$ is given by
	\begin{equation}
		\bar{i}(x, y) = |x - y| \enspace.
	\end{equation}
	The \textbf{unordered pitch-class interval}, or simply interval class, between
	$x$ and $y$ is given by
	\begin{equation}
		\bar{i}(\bar{x}, \bar{y}) = \min\{i(\bar{x}, \bar{y}), i(\bar{y}, \bar{x})\} \enspace.
	\end{equation}
\end{definition}

%--------------------------------------------------------------------------
\begin{example}
	Put $x = 43$ and $y = -13$. Then
	\begin{gather}
		i(x, y) = -56 \\
		i(\bar{x}, \bar{y}) = \overline{11} - \bar{7} = \bar{4} \\
		\bar{i}(x, y) = 56 \\
		\bar{i}(\bar{x}, \bar{y}) = \bar{4}
	\end{gather}
	Whenever the context is clear, we shall drop quotient notation and subscripts.
	In most situations, we are interested in the interval class between $x$ and $y$,
	in which case we will simply write $i(7, -1) = 4$. Interval classes can also be
	seen graph-theoretically as the edge connecting two members of a pitch-class set,
	displayed clockwise.
\end{example}

%--------------------------------------------------------------------------
\begin{theorem}
	\label{rahn-common-tone}
	\cite[10]{Rahn1975}
	The number of common tones between a set $S$ and some transposition of itself is given by
	\begin{equation}
		|S \cap \T_n(S)| = |\{x - y = n : x, y \in S\}| \enspace.
	\end{equation}
	The number of common tones between a set $S$ and some inversion of itself is given by
	\begin{equation}
		|S \cap \T_n\I(S)| = 2 \cdot |\{x + y = n : x, y \in S\}| +
		|\{a \in S : 2a = n\}| \enspace.
	\end{equation}
	Moreover, the cardinality of the set $\{a \in S : 2a = n\}$ is at most 2.
	\begin{proof}
		We must count the occurrences of pairs of pitch classes that are interchanged by
		the operation at hand and double them, for if $x$ maps onto $y$ under some $\T_n\I$,
		then certainly $y$ maps onto $x$ under the same operation, given that every inversion
		operation has order two. In addition to that, we must account for the occurrences of
		pitch classes that may map onto themselves under the aforementioned operation. For
		any pair $a \ne b \in S$, it follows $a$ and $b$ are exchanged by some operation
		$\T_n\I$ whenever both $\T_n\I(a) = b$ and $\T_n\I(b) = a$ hold. Since
		$\T_n\I(a) = -a + n$ and similarly $\T_n\I(b) = -b + n$, if the pair is exchanged,
		we must have $-a + n = b$ and $-b + n = a$ both true. Adding the last two
		expressions and yields $a + b = n$, which is the first set in the right-hand side
		of the formula. As discussed above, the cardinality of this set must be doubled.
		We have for any $a$ that $\T_n\I(a) = a + n$, hence $a = \T_n\I(a) \iff a = -a + n$,
		that is, whenever $2a = n$. That is the second set in the formula. Finally, for
		any pair $(a, n)$ such that $a = \T_n\I(a)$, we also have
		$a + 6 = -(a + 6) + n \iff 2a = n$, so that by the above it follows
		$a + 6 = \T_n\I(a + 6)$. Thus the set $\{a \in S : 2a = n\}$ has cardinality
		at most 2, proving the last assertion.
	\end{proof}
\end{theorem}

%--------------------------------------------------------------------------
\begin{example}
	\label{rahn-example}
	\cite[11]{Rahn1975}
	Write $S = \{ 0, 1, 4, 5, 8, 9 \}$ and consider some inversion operation. We have
	\begin{equation}
		\begin{array}{ *{13}{c} }
			n & 0 & 1 & 2 & 3 & 4 & 5 & 6 & 7 & 8 & 9 & 10 & 11 \\
			2 \cdot |\{x + y = n : x, y \in S\}| &
			2 & 6 & 2 & 0 & 2 & 6 & 2 & 0 & 2 & 6 & 2 & 0 \\
			|\{a \in S : 2a = n\}| & 1 & 0 & 1 & 0 & 1 & 0 & 1 & 0 & 1 & 0 & 1 & 0 \\
			Total & 3 & 6 & 3 & 0 & 3 & 6 & 3 & 0 & 3 & 6 & 3 & 0
		\end{array}
	\end{equation}
\end{example}

%--------------------------------------------------------------------------
\begin{example}
	We can demonstrate \ref{rahn-example}, as well as the omitted proof of
	\ref{rahn-common-tone} under transposition in a much simpler way with a little bit
	of abstract algebra. By observing the cycle decomposition the each operation at hand,
	if $n = 3$, then we have
	\begin{equation}
		\T_3\I = (0 \; 3) (1 \; 2) (4 \; 11) (5 \; 10) (6 \; 9) (7 \; 8) \enspace.
	\end{equation}
	Hence, under $\T_3\I$, every pitch-class in $S = \{ 0, 1, 4, 5, 8, 9 \}$ maps to
	the complement of $S$. If the operation is, for instance, $\T_9$, then since
	\begin{equation}
		\T_9 = (0 \; 9 \; 6 \; 3) (1 \; 10 \; 7 \; 4) (2 \; 11 \; 8 \; 5) \enspace,
	\end{equation}
	we get straightforwardly that $S = \{ 0, 1, 4, 5, 8, 9 \}$ shares three common tones
	with $\T_9 \circ S$, namely $0 \mapsto 9$, $4 \mapsto 1$, and $8 \mapsto 5$.
\end{example}

%--------------------------------------------------------------------------
\section{ELEMENTARY GROUP THEORY}

% TODO Abstract and permutation groups

%--------------------------------------------------------------------------
\begin{proposition}
	\cite[127]{FripertingerLackner2015}
	We have the following isomorphisms:
	\begin{gather}
		\langle \T \rangle \cong C_{12} \\
		\langle \T, \I \rangle \cong D_{24} \\
		\langle \I, \R \rangle \cong V_4 \\
		\langle \T, \I, \R \rangle \cong D_{24} \oplus \mathbb{F}_2 \\
		\langle \T, \I, \M \rangle \cong \Aff_1(C_{12}) \\
		\langle \T, \I, \M, \R \rangle \cong \Aff_1(C_{12}) \oplus \mathbb{F}_2
	\end{gather}
\end{proposition}
