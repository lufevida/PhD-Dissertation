
%--------------------------------------------------------------------------
\chapter{}

%--------------------------------------------------------------------------
\section{ELEMENTARY TWELVE-TONE THEORY}

\begin{definition}
	Let $x$ and $y$ be arbitrary pitches. The \textbf{ordered pitch interval} between
	$x$ and $y$ is given by
	\begin{equation}
		i(x, y) = y - x \enspace.
	\end{equation}
	The \textbf{ordered pitch-class interval} between $x$ and $y$ is given by
	\begin{equation}
		i(\bar{x}, \bar{y}) = \bar{y} - \bar{x} \enspace.
	\end{equation}
	The \textbf{unordered pitch interval} between $x$ and $y$ is given by
	\begin{equation}
		\bar{i}(x, y) = |x - y| \enspace.
	\end{equation}
	The \textbf{unordered pitch-class interval}, or simply interval class, between
	$x$ and $y$ is given by
	\begin{equation}
		\bar{i}(\bar{x}, \bar{y}) = \min\{i(\bar{x}, \bar{y}), i(\bar{y}, \bar{x})\} \enspace.
	\end{equation}
\end{definition}

\begin{example}
	Put $x = 43$ and $y = -13$. Then
	\begin{gather}
		i(x, y) = -56 \\
		i(\bar{x}, \bar{y}) = \overline{11} - \bar{7} = \bar{4} \\
		\bar{i}(x, y) = 56 \\
		\bar{i}(\bar{x}, \bar{y}) = \bar{4}
	\end{gather}
	Whenever the context is clear, we shall drop quotient notation and subscripts.
	In most situations, we are interested in the interval class between $x$ and $y$,
	in which case we will simply write $i(7, -1) = 4$. Interval classes can also be
	seen graph-theoretically as the edge connecting two members of a pitch-class set,
	displayed clockwise.
\end{example}

% TODO Interval classes are equivalence relations
