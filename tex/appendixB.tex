\chapter{ELEMENTARY GROUP THEORY}

%\clearpage
%\thispagestyle{plain}
%\begin{landscape}
%\begin{figure}
%\begin{center}
%\includegraphics[width=6in]{images/LaTeX2e_logo.eps}
%\caption{\LaTeX 2\ensuremath{\epsilon.} logo}\label{biglogo}
%\end{center}
%\end{figure}
%\end{landscape}

\section{These are all the examples}

\ensuremath{\epsilon}

\begin{figure}[htbp]
\centering
\includegraphics[width=5in]{images/diagram} %[scale=1.0]
\caption[]{}
\end{figure}

\begin{table}[htbp]
\caption{A sample Table}\label{first}
\begin{tabularx}{6.5in}{XXX}
\hline
First & Second & Third \\
\hline
12 & 45 & 26 \\
17 & 32 & 93 \\
text & 51 & can be there too. \\	
\epsfig{figure=images/cat.eps, scale=1} & 28 & Figures too - a cat. \\
\begin{turn}{0}\epsfig{figure=images/mouse.eps, scale=0.25}\end{turn} & 000 & and a mouse! \\
\hline
\end{tabularx}
\end{table}

\begin{table}
\caption{Feasible triples for highly variable Grid, MLMMH.} \label{tbl1}
\begin{tabularx}{6.5 in}{r l X}
\hline {{Time (s)}} & {{Triple chosen}} & {{Other feasible triples}} \\ \hline
0 & (1, 11, 13725) & (1, 12, 10980), (1, 13, 8235), (2, 2, 0), (3, 1, 0) \\
2745 & (1, 12, 10980) & (1, 13, 8235), (2, 2, 0), (2, 3, 0), (3, 1, 0) \\
5490 & (1, 12, 13725) & (2, 2, 2745), (2, 3, 0), (3, 1, 0) \\
\hline
\end{tabularx}
\end{table}

\begin{table}[h!t!]
\begin{tabularx}{6.5 in}{r l X}
\multicolumn{3}{l}{Table \ref{tbl1}. Continued}\\%
\hline {{Time (s)}} & {{Triple chosen}} & {{Other feasible triples}} \\ \hline
115290 & (1, 13, 16470) & (2, 2, 2745), (2, 3, 0), (3, 1, 0) \\
118035 & (1, 13, 13725) & (2, 2, 2745), (2, 3, 0), (3, 1, 0) \\
120780 & (1, 13, 16470) & (2, 2, 2745), (2, 3, 0), (3, 1, 0) \\
123525 & (1, 13, 13725) & (2, 2, 2745), (2, 3, 0), (3, 1, 0) \\
\hline
\end{tabularx}
\end{table}
