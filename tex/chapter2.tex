%--------------------------------------------------------------------------
\chapter{THEORETICAL FRAMEWORK}

%--------------------------------------------------------------------------
\section{Group Actions}

%--------------------------------------------------------------------------
\begin{definition}
	\cite[99]{Rotman1967}
	\cite[41]{DummitFoote2004}
	Let $X$ be a set and $G$ a group. An \textbf{action} of $G$ on $X$ is a function
	$G \times X \to X$ given by $(g, x) \mapsto gx$, such that:
	\begin{enumerate}[i.]
		\item $(gh)x = g(hx)$ for all $g, h \in G$ and $x \in X$;
		\item $1x = x$ for all $x \in X$, where $1 \in G$ is the identity.
	\end{enumerate}
\end{definition}

%--------------------------------------------------------------------------
\begin{proposition}
	\cite[99]{Rotman1967}
	\cite[42]{DummitFoote2004}
	If a group $G$ acts on a set $X$ then, for every $g \in G$, the function
	$f_g : X \to X$ given by $f_g(x) = gx$ is a permutation of $X$. Further, the function
	$f : G \to S_X$ given by $f(g) = f_g$ is a homomorphism and, conversely, for any
	homomorphism $\phi : G \to S_X$, there is a corresponding group action given by
	$\phi(g)(x)$.
\end{proposition}

%--------------------------------------------------------------------------
\begin{theorem}[Cayley]
	\cite[96]{Rotman1967}
	\cite[120]{DummitFoote2004}
	Every group is isomorphic to a subgroup of the symmetric group $S_G$. In particular,
	if $|G| = n$, then $G$ is isomorphic to a subgroup of $S_n$.
\end{theorem}

%--------------------------------------------------------------------------
\begin{theorem}
	\cite[97]{Rotman1967}
	Let $H \leq G$ be a subgroup of finite index $n$. Then there exists a homomorphism
	$\phi : G \to S_n$ such that $\ker \phi \leq H$. In particular, when $H = \{ 1 \}$,
	we get Cayley's theorem.
\end{theorem}

%--------------------------------------------------------------------------
\begin{example}
	\cite[122]{DummitFoote2004}
	A group acts on itself by conjugation. Let $g, h \in G$ and $x \in G$. Then
	$1 x 1^{-1} = x$ and
	\begin{equation}
		g(hx) = g(h x h^{-1}) = gh x h^{-1}g^{-1} = (gh) x (gh)^{-1} = (gh)x \enspace.
	\end{equation}
	It is also immediate from the above that a group acts on its power set by conjugation.
	In particular, a group acts on the set of all its subgroups.
\end{example}

%--------------------------------------------------------------------------
\begin{definition}
	\cite[100]{Rotman1967}
	\cite[112]{DummitFoote2004}
	If $G$ acts on $X$, then the \textbf{orbit} of $x \in X$ is the set
	\begin{equation}
		\mathcal{O}(x) = \{ gx : g \in G \} \subseteq X \enspace.
	\end{equation}
	We say an action is \textbf{transitive} if there is only one orbit. The \textbf{kernel}
	of the action is the set
	\begin{equation}
		\{ g \in g : gx = x, \forall x \in X \} \enspace.
	\end{equation}
	We say an action is \textbf{faithful} if the kernel is the identity. The
	\textbf{stabilizer} of $x$ in $G$ is the group
	\begin{equation}
		G_x = \{ g \in g : gx = x \} \leq G \enspace.
	\end{equation}
	When a group acts on itself by conjugation, we call the orbits
	\textbf{conjugacy classes}. The stabilizer of some $g \in G$ is the
	\textbf{centralizer} of $g$ in $G$, denoted $C_G(g)$. When a group acts on the set of
	its subgroups by conjugation, the stabilizer of a subgroup $H \leq$ is the
	\textbf{normalizer} of $H$ in $G$, denoted by $N_G(H)$.
\end{definition}

%--------------------------------------------------------------------------
\begin{proposition}
	\cite[102]{Rotman1967}
	\cite[114]{DummitFoote2004}
	\cite[250]{Tucker1974}
	If $G$ acts on $X$, for $x_1, x_2 \in X$, the relation $x_1 \sim x_2$ given by
	$x_1 = gx_2$ is an equivalence relation. It follows immediately that the equivalence
	classes are the orbits of the action of $G$ on $X$ and that
	\begin{equation}
		|X| = \sum_i |\mathcal{O}(x_i)| \enspace,
	\end{equation}
	where $x_i$ is a single representative from each orbit.
\end{proposition}

%--------------------------------------------------------------------------
\begin{theorem}[Orbit-Stabilizer]
	\cite[102]{Rotman1967}
	If $G$ acts on $X$, then for each $x \in X$
	\begin{equation}
		|\mathcal{O}(x)| = [G : G_x] \enspace.
	\end{equation}
\end{theorem}

%--------------------------------------------------------------------------
\begin{corollary}
	\cite[103]{Rotman1967}
	If $G$ is finite and acts on $X$, then the size of any orbit is a divisor of $|G|$.
\end{corollary}

%--------------------------------------------------------------------------
\begin{proposition}
	\cite[123]{DummitFoote2004}
	The number of conjugates of a subset $S$ in a group $G$ is $|G : N_G(S)|$, the index
	of the normalizer of $S$. In particular, the number of conjugates of an element $s$ is
	$|G : C_G(s)|$, the index of the centralizer of $s$.
\end{proposition}

%--------------------------------------------------------------------------
\begin{proposition}
	\cite[125]{DummitFoote2004}
	Let $\sigma, \tau \in S_n$. If
	\begin{equation}
		\sigma = (a_1 \; a_2 \; \cdots a_j) (b_1 \; b_2 \; \cdots a_k) \cdots \enspace,
	\end{equation}
	then
	\begin{equation}
		\tau \sigma \tau^{-1} = (\tau(a_1) \; \tau(a_2) \; \cdots
		\tau(a_j)) (\tau(b_1) \; \tau(b_2) \; \cdots \tau(b_k)) \cdots \enspace.
	\end{equation}
\end{proposition}

%--------------------------------------------------------------------------
\begin{example}
	\cite[127]{DummitFoote2004}
	Let $\sigma \in S_n$ be an $m$-cycle. The number of conjugates of $\sigma$ is
	\begin{equation}
		\frac{|S_n|}{|C_{S_n}(\sigma)|} = \frac{n (n - 1) (n - m + 1)}{m} \enspace,
	\end{equation}
	so that $|C_{S_n}(\sigma)| = m (n - m)!$. Since $\sigma$ commutes with its powers,
	and also with any permutation in $S_n$ whose cycles are disjoint from it, and there are
	$(n - m)!$ of those, the number computed above is the full centralizer of $\sigma$.
\end{example}

%--------------------------------------------------------------------------
\begin{example}
	\cite[132]{DummitFoote2004}
	The size of each conjugacy class in $S_n$ is
	\begin{equation}
		\frac{n!}{\prod_{r}r^{n_r}n_r!} \enspace,
	\end{equation}
	where, for each $r$-cycle, we divide by $r$ to account for the cyclical permutations
	of elements within a cycle. Further, if there are $n_r$ cycles of length $r$, we divide
	by $n_r!$ to account for the different orders in which those cycles may appear.
\end{example}

%--------------------------------------------------------------------------
\begin{proposition}
	\cite[126]{DummitFoote2004}
	Two elements in $S_n$ are conjugate if and only if they have the same cycle type.
	The number of conjugacy classes of $S_n$ is the number of partitions of $n$.
\end{proposition}

%--------------------------------------------------------------------------
\begin{definition}
	\cite[133]{DummitFoote2004}
	An isomorphism from a group $G$ to itself is called an \textbf{automorphism} of $G$.
	The group under composition of all automorphisms of $G$ is denoted by $\Aut(G)$.
\end{definition}

%--------------------------------------------------------------------------
\begin{proposition}
	\cite[133]{DummitFoote2004}
	If $H$ is a normal subgroup of $G$, then the action of $G$
	by conjugation on $H$ is, for each $g \in G$, an automorphism of $H$. The kernel of the
	action is $C_G(H)$. In particular, $G / c_G(H)$ is a subgroup of $\Aut(H)$.
\end{proposition}

%--------------------------------------------------------------------------
\begin{corollary}
	\cite[134]{DummitFoote2004}
	For any subgroup $H < G$ and $g \in G$, $H \cong gHg^{-1}$. Moreover, $N_G(H) / C_G(H)$
	(and also $G/Z(G)$ when $G = H$) is isomorphic to a subgroup of $\Aut(H)$.
\end{corollary}

%--------------------------------------------------------------------------
\begin{definition}
	\cite[135]{DummitFoote2004}
	A subgroup $H < G$ is called \textbf{characteristic} if every automorphism of $G$ maps
	$H$ to itself.
\end{definition}

%--------------------------------------------------------------------------
\begin{proposition}
	\cite[135]{DummitFoote2004}
	Characteristic subgroups are normal. Unique subgroups of a given order are
	characteristic. A characteristic subgroup of a normal subgroup is normal. In particular,
	every subgroup of a cyclic group is characteristic.
\end{proposition}

%--------------------------------------------------------------------------
\begin{proposition}
	\cite[135]{DummitFoote2004}
	If $G$ is cyclic of order $n$, then $\Aut(G) \cong (\mathbb{Z} / n \mathbb{Z})^\times$,
	and $|\Aut(G)| = \varphi(n)$, where $\varphi$ is Euler's totient function.
\end{proposition}

%--------------------------------------------------------------------------
\begin{corollary}
	\cite[136]{DummitFoote2004}
	Let $|G| = pq$, with $p \leq q$ primes. If $p \nmid q - 1$, then $G$ is abelian.
	If, further, $p < q$, then $G$ is cyclic.
\end{corollary}

%--------------------------------------------------------------------------
\begin{proposition}
	\cite[136]{DummitFoote2004}
	If $|G| = p^n$, with $p$ an odd prime, then
	\begin{equation}
		|\Aut(G)| = p^{n - 1} (p - 1)
	\end{equation}
	and the latter group is also cyclic. If $|G| = 2^n$ is cyclic, with $n \geq 3$, then
	\begin{equation}
		\Aut(G) \cong \mathbb{Z} / 2 \mathbb{Z} \times \mathbb{Z} / 2^{n - 2} \mathbb{Z}
	\end{equation}
	and the latter is not cyclic, but has a cyclic subgroup of index 2. If $V$ is the
	elementary abelian group of order $p^n$, then $pv = 0$ for all $v \in V$ and $V$ is an
	$n$-dimensional vector space over the field $\mathbb{F}_p = \mathbb{Z} / p \mathbb{Z}$.
	The automorphisms of $V$ are the nonsingular linear transformations from $V$ to itself:
	\begin{equation}
		\Aut(V) \cong GL(V) \cong GL_n(\mathbb{F}_p) \enspace,
	\end{equation}
	and if $F$ is the finite field of order $q$, then
	\begin{equation}
		|GL_n(F)| = (q^n - 1) (q^n - q) (q^n - q^2) \cdots (q^n - q^{n - 1}) \enspace.
	\end{equation}
	For all $n \ne 6$, we have $\Aut(S_n) \cong S_n$. Finally, $\Aut(D_8) \cong D_8$ and
	$\Aut(Q_8) \cong S_4$.
\end{proposition}

%--------------------------------------------------------------------------
\begin{example}
	\cite[137]{DummitFoote2004}
	The Klein 4-group is the elementary abelian group of order 4. It follows
	$\Aut(V_4) \cong GL_2(\mathbb{F}_2)$, and $|\Aut(V_4)| = 6$. Since the action of
	$\Aut(V_4)$ on $V_4$ permutes the latter's 3 nonidentity elements, by order
	considerations we have
	\begin{equation}
		\Aut(V_4) \cong GL_2(\mathbb{F}_2) \cong S_3 \enspace.
	\end{equation}
\end{example}

%--------------------------------------------------------------------------
\begin{proposition}
	\cite[314]{DummitFoote2004}
	A finite subgroup of the multiplicative group of a field is cyclic. In particular,
	if $F$ is a finite field, then $F^\times$ is cyclic.
\end{proposition}

%--------------------------------------------------------------------------
\begin{corollary}
	\cite[314]{DummitFoote2004}
	For $p$ a prime, $(\mathbb{Z} / p \mathbb{Z})^\times$ is cyclic.
\end{corollary}

%--------------------------------------------------------------------------
\begin{proposition}
	\label{morris-mallalieu}
	\cite[17]{Morris1976}
	A 12-tone row has the mallalieu property if and only if it is related by $\T_n \M \I$
	to the row $S = \{ 0, 1, 4, 2, 9, 5, 11, 3, 8, 10, 7, 6 \}$.
	\begin{proof}
		One direction is just the straightforward check that every $\T_n \M \I$ transform of
		$S$ possesses the mallalieu property and is left to the reader. Conversely, if a
		row $R$ in its untransposed prime form has the mallalieu property, then there is
		a transposition that takes its order numbers in zeroth rotation, that is, the set
		$\{ 0, 1, 2, \dots, 11 \}$ to its order numbers in, say, first rotation, id est,
		the set $\{ 1, 3, \dots, 11, 0, 2, \dots, 10 \}$. We can write this transposition
		as a permutation $0 \mapsto 1, 1 \mapsto 3, \dots, 11 \mapsto 10 $, or in cycle
		notation as $T_k = ( 0 \; 1 \; 3 \; 7 \; 2 \; 5 \; 11 \; 10 \; 8 \; 4 \; 9 \; 6 )$.
		Note that $\T_k$ is an operation on order numbers. Since $\T_k$ is a transposition,
		there are only four candidates for $k$, namely $k \in \{ 1, 5, 7, 11 \}$
		(because these are the only indices for which a transposition in cycle notation is
		a 12-cycle). Moreover, we do not need to consider the cases where
		$k \in \{5, 7, 11\}$, as $\T_5 = \M \circ \T_1$, $\T_7 = \M \I \circ \T_1$,
		and $\T_{11} = \I \circ \T_1$. Hence, without loss, we can set $k = 1$. But then
		$S$ is the only row in untransposed prime form where $\T_1$ induces the permutation
		$\T_k$ from its order numbers in zeroth rotation to its order numbers in first
		rotation (just equate $\T_k$ with $\T_1$), completing the proof.
	\end{proof}
\end{proposition}

%--------------------------------------------------------------------------
\begin{proposition}
	\label{lewin-mallalieu}
	\cite[285]{Lewin1966}
	For $p$ a prime, every $(p - 1)$-TET system is capable of producing a mallalieu row.
	\begin{proof}
		For every prime $p$, the group of units modulo $p$ is isomorphic to
		$\mathbb{Z} / (p - 1) \mathbb{Z}$. The mallalieu property in these cases can be
		seen as the aforementioned isomorphism, where $\mathbb{Z} / (p - 1) \mathbb{Z}$
		is the group of transpositions of a row, and $(\mathbb{Z} / p \mathbb{Z})^\times$
		is its multiplicative group on order numbers. The number of mallalieu rows in each
		$(p - 1)$-TET system is then the number of isomorphisms
		$\mathbb{Z} / (p - 1) \mathbb{Z} \to (\mathbb{Z} / p \mathbb{Z})^\times$, that is,
		the order of the group of automorphisms of $\mathbb{Z} / (p - 1) \mathbb{Z}$.
		Since for every prime $p$ we have $|\Aut(\mathbb{Z} / (p - 1) \mathbb{Z})| \geq 1$,
		every $(p - 1)$-TET system is capable of producing a mallalieu row, as desired.
	\end{proof}
\end{proposition}

%--------------------------------------------------------------------------
\begin{example}
	\cite[8]{Lewin1976a}
	\cite[9]{Babbitt1976}
	In face of \ref{lewin-mallalieu}, \ref{morris-mallalieu} becomes just the special
	case where $p = 13$. The number of isomorphisms
	$\mathbb{Z} / 12 \mathbb{Z} \to (\mathbb{Z} / 13 \mathbb{Z})^\times$ is equal to
	\begin{equation}
		|\Aut\big((\mathbb{Z} / 12 \mathbb{Z})^\times\big)| = 4 \enspace.
	\end{equation}
	We can construct these isomorphisms by mapping a generator of
	$\mathbb{Z} / 12 \mathbb{Z}$, say $\bar{1}$, to the generators of
	$(\mathbb{Z} / 13 \mathbb{Z})^\times$, namely $\bar{2}, \bar{6}, \bar{7}$ and
	$\overline{11}$. Explicitly, we get the four maps $i \pmod{12} \mapsto 2^i \pmod{13}$,
	$i \pmod{12} \mapsto 6^i \pmod{13}$, $i \pmod{12} \mapsto 7^i \pmod{13}$, and
	$i \pmod{12} \mapsto 11^i \pmod{13}$. We leave the verification that these maps are
	well defined and bijective to the reader. Denote the first map by $\varphi$. Then
	\begin{equation}
		\varphi(a + b) = 2^{a + b} = 2^a \cdot 2^b = \varphi(a) \cdot \varphi(b) \enspace,
	\end{equation}
	so $\varphi$ is an isomorphism. The verification that the other three maps are
	isomorphisms is identical. Define
	$\varphi^{-1} : (\mathbb{Z} / 13 \mathbb{Z})^\times \to \mathbb{Z} / 12 \mathbb{Z}$ by
	$\varphi^{-1}(\log i \pmod{13}) = i \pmod{12}$. Then $\varphi^{-1}$ is easily seen to
	be the inverse of $\varphi$. Let $S^* = \{ 1, 2, \dots, 12 \}$ be a series of order
	numbers written multiplicatively. Then
	\begin{equation}
		\varphi^{-1}(S^*) = \{ \log 1, \log 2, \dots, \log 12 \} \pmod{13} =
		\{ 0, 1, 4, \dots, 7, 6 \} \enspace,
	\end{equation}
	which by \ref{morris-mallalieu} is one of the four 12-tone rows with the mallalieu
	property.
\end{example}

%--------------------------------------------------------------------------
\section{Polya's Enumeration Formula}

%--------------------------------------------------------------------------
\begin{definition}
	\cite[85]{Aigner2007}
	Let $N$ be a set of $n$ beads and $R$ be a set of $r$ colors. A colored necklace is
	a function $f : N \to R$. Denote the set of all such functions by $R^N$, so that
	$|R^N| = r^n$. Define the \textbf{weight} associated to a coloring $f$ by
	\begin{equation}
		w(f) = \prod_{i \in N} x_{f(i)} \enspace,
	\end{equation}
	where $x_j$ is a variable associated with the color $j \in R$.
\end{definition}

%--------------------------------------------------------------------------
\begin{proposition}
	\label{coloring}
	\cite[110]{Rotman1967}
	Let $G$ be a group and $X = \{ 1, \cdots, n \}$ be a set. Let $\mathcal{C}$ be a set
	of $q$ colors. Then $G$ acts on the set $\mathcal{C}^n$ of $n$-tuples of colors by
	\begin{equation}
		\tau(c_1, \cdots, c_n) = (c_{\tau 1}, \cdots, c_{\tau n}),
		\forall \tau \in G \enspace.
	\end{equation}
\end{proposition}

%--------------------------------------------------------------------------
\begin{proposition}
	\cite[85]{Aigner2007}
	\cite[110]{Rotman1967}
	If $G$ is a group acting on the set $N$, then two colorings $f$ and $f^\prime$
	are equivalent whenever $f = f^\prime \circ g$ for some $g \in G$. This is an
	equivalence relation that partitions $R^N$ into equivalence classes denoted
	\textbf{patterns}. Under the conditions of \ref{coloring}, an orbit
	$(c_1, \cdots, c_n) \in \mathcal{C}^n$ is called a $(q, G)$-\textbf{coloring} of $X$.
	Moreover, two equivalent colorings have the same weight, so that we may refer to
	the weight of a class of colorings rather than the weights of its representatives.
\end{proposition}

%--------------------------------------------------------------------------
\begin{definition}
	\cite[85]{Aigner2007}
	Let $N$ be a set of $n$ beads and $R$ be a set of $r$ colors. Let $G$ be a group acting
	on the set $N$ and $x_j$ the variable associated with the color $j \in R$. Let
	$\mathcal{M}$ be the set of of patterns. Define the pattern \textbf{enumerator} by
	\begin{equation}
		w(R^N, G) = \sum_{M \in \mathcal{M}} w(M) \enspace.
	\end{equation}
	In particular, when $x_j = 1$ for all $j \in R$, we get $w(R^N, G) = |\mathcal{M}|$.
\end{definition}

%--------------------------------------------------------------------------
\begin{definition}
	\cite[86]{Aigner2007}
	Let $G$ be a group acting on a set $X$. Define for every $g \in G$ its
	\textbf{fixed-point set} by 
	\begin{equation}
		\Fix(g) = \{ x \in X : gx = x \} \enspace.
	\end{equation}
\end{definition}

%--------------------------------------------------------------------------
\begin{lemma}
	\cite[112]{Rotman1967}
	Let $G < S_n$ be a group and let $\mathcal{C}$ be a set of $q$ colors. For $\tau \in G$,
	\begin{equation}
		|\Fix(\tau)| = q^{t(\tau)} \enspace,
	\end{equation}
	where $t(\tau)$ is the number of cycles in the complete factorization of $\tau$.
\end{lemma}

%--------------------------------------------------------------------------
\begin{lemma}[Burnside]
	\cite[109]{Rotman1967}
	\cite[251]{Tucker1974}
	If $G$ acts on a finite set $X$, then the number of orbits $N$ is
	\begin{equation}
		N = \frac{1}{|G|} \sum_{\tau \in G} |\Fix(\tau)| \enspace.
	\end{equation}
\end{lemma}

%--------------------------------------------------------------------------
\begin{corollary}
	\cite[112]{Rotman1967}
	Let $G$ be a group acting on a finite set $X$. The number $N$ of $(q, G)$-colorings
	of $X$ is
	\begin{equation}
		N = \frac{1}{|G|} \sum_{\tau \in G} q^{t(\tau)} \enspace.
	\end{equation}
\end{corollary}

%--------------------------------------------------------------------------
\begin{corollary}
	\cite[54]{Reiner1985}
	\cite[127]{FripertingerLackner2015}
	The number of $n$-tone rows under $\R \T_n \I$ is
	\begin{equation}
		\begin{cases}
			\frac{1}{4} \left[ (n - 1)! + (n - 1) (n - 3) \cdots (2) \right]
			& n \text{ odd} \\
			\frac{1}{4} \left[ (n - 1)! + (n - 2) (n - 4) \cdots (2) (1 + \frac{n}{2})
			\right] & n \text{ even} \\
		\end{cases}
	\end{equation}
	In particular, there are 9985920 twelve-tone rows.
\end{corollary}

%--------------------------------------------------------------------------
\begin{definition}
	\cite[87]{Aigner2007}
	Let $g$ be a group acting on $R^N$. Define the \textbf{cycle indicator} of $G$ by
	\begin{equation}
		P_G(z_1, z_2, \cdots, z_n) = \frac{1}{|G|} \sum_{g \in G} z_1^{b_1(g)} z_2^{b_2(g)}
		\cdots z_n^{b_n(g)} \enspace,
	\end{equation}
	where $z_i^{b_i(g)}$ corresponds to the number of cycles in the complete factorization
	of $g$ that have length $i$.
\end{definition}

%--------------------------------------------------------------------------
\begin{example}
	The cycle indicator of $S_3$ is
	\begin{equation}
		P_{S_3}(x_1, x_2, x_3) = \frac{1}{6}(x_1^3 + 3 x_1^1 x_2^1 + 2 x_3^1) \enspace,
	\end{equation}
	since there are two elements in $S_3$ that comprise one cycle of length three,
	three elements that comprise one cycle of length one, and one cycle of length two,
	and one element that comprises three cycles of length one.
\end{example}

%--------------------------------------------------------------------------
\begin{example}
	Let $C_n$ be the cyclic group of order $n$. Then
	\begin{equation}
		P_{C_n} = \frac{1}{n} \sum_{d | n} \varphi(d) z_d^{n / d} \enspace.
	\end{equation}
\end{example}

%--------------------------------------------------------------------------
\begin{example}
	For $S_n$ we have
	\begin{equation}
		P_{S_n} = \sum_{j_1 + 2j_2 + \cdots + nj_n = n}
		\frac{1}{\prod_{k = 1}^n k^{j_k} j_k!} \prod_{k = 1}^n a_k^{j_k} \enspace.
	\end{equation}
	In words, there is a summand for each conjugacy class in $S_n$, and we divide each
	summand by the size of its corresponding conjugacy class.
\end{example}

%--------------------------------------------------------------------------
\begin{example}
	For $D_n$ we have
	\begin{equation}
		P_{D_n} = \frac{1}{2} P_{C_n} +
		\begin{cases}
			\frac{1}{2} a_1 a_2^{(n - 1) / 2} & n \text{ odd} \\
			\frac{1}{4} (a_1^2 a_2^{(n - 2) / 2} + a_2^{n / 2)} & n \text{ even}
		\end{cases}
	\end{equation}
	In particular, we obtain
	\begin{equation}
		P_{D_{24}}(1 + x, \cdots, 1 + x^{12}) = \frac{1}{24} ( x_1^{12} + 6 x_1^2 x_2^5 +
		7 x_2^6 + 2 x_3^4 + 2 x_4^3 + 2 x_6^2 + 4 x_{12} ) \enspace.
	\end{equation}
\end{example}

%--------------------------------------------------------------------------
\begin{example}
	%\cite[1]{WeiXu1993}
	\cite[120]{FripertingerLackner2015}
	For $\Aff_1(C_{12})$ we have
	\begin{multline}
		P_{\Aff_1(C_{12})}(1 + x, \cdots, 1 + x^{12}) = \frac{1}{48} (
		x_1^{12} + 2 x_1^6 x_2^3 + 3 x_1^4 x_2^4 + 6 x_1^2 x_2^5 \\ + 12 x_2^6 +
		4 x_2^3 x_6 + 2 x_3^4 + 8 x_4^3 + 6 x_6^2 + 4 x_{12} ) \enspace.
	\end{multline}
\end{example}

%--------------------------------------------------------------------------
\begin{theorem}[Polya]
	\cite[88]{Aigner2007}
	\cite[256]{Tucker1974}
	Let $N$ be a set of cardinality $n$ and $R$ be a set of cardinality $r$. Let $G$ be
	a group acting on the set $N$ and $x_j$ be arbitrary variables with $j \in R$.
	Let $w(R^N, G)$ be the pattern enumerator for the action of $G$ on $R^N$. Then
	\begin{equation}
		w(R^N, G) = \sum_{M \in \mathcal{M}} w(M) = P_G(\sum_{j \in R} x_j,
		\sum_{j \in R} x_j^2, \cdots, \sum_{j \in R} x_j^n) \enspace.
	\end{equation}
\end{theorem}

%--------------------------------------------------------------------------
\begin{corollary}
	\cite[89]{Aigner2007}
	\cite[254]{Tucker1974}
	Let $x_j = 1$ for all $j \in R$. Then $\sum_{j \in R} x_j^k = |R| = r$ for all $k$, hence
	\begin{equation}
		P_G(r, \cdots, r) = |\mathcal{M}| \enspace.
	\end{equation}
\end{corollary}

%--------------------------------------------------------------------------
\begin{corollary}
	\label{oneColor}
	\cite[89]{Aigner2007}
	Let $|R| = r = 2$. Let $x_1 = x_{\text{white}} = x$ and
	$x_2 = x_{\text{black}} = 1$. Then
	\begin{equation}
		P_G(x + 1, x^2 + 1, \cdots, x^n + 1) = \sum_{k = 0}^n a_k x^k \enspace,
	\end{equation}
	where $a_k$ is the number of patterns in which the color white occurs exactly $k$ times.
\end{corollary}

%--------------------------------------------------------------------------
\begin{example}
	\cite[89]{Aigner2007}
	The number of necklaces with $n$ beads and $r$ colors is
	\begin{equation}
		\frac{1}{n} \sum_{d | n} \varphi(d) r_d^{n / d} \enspace.
	\end{equation}
\end{example}

%--------------------------------------------------------------------------
\begin{example}
	\cite[86]{Aigner2007}
	The definition of a weight function is useful in order to count how many necklaces
	contain precisely $x_j$ beads of color $j$. Set $W$ to be the color white and $B$
	to be the color black. Then a 4-bead necklace with exactly 2 white and 2 black beads
	is expressed by $W^2B^2$. If we do not wish to account for a particular color,
	per \ref{oneColor} we may assign its weight to 1. If we chose not to account for the color
	black, say, then the representation above would become simply $W^21^2 = W^2$.
\end{example}

%--------------------------------------------------------------------------
\begin{example}
	We use Polya's theorem to describe the pattern inventory of 3-bead necklaces under
	the action of $S_3$. Let the colors be $A$ and $B$. Then
	\begin{equation}
		\begin{align}
			P_{S_3}(A + B, A^2 + B^2, A^3 + B^3)
			& = \frac{1}{6}[(A + B)^3 + 3 (A + B) (A^2 + B^2) + 2 (A^3 + B^3)] \\
			& = A^3 + A^2 B + A B^2 + B^3 \enspace.
		\end{align}
	\end{equation}
	In words, we have one necklace with three $A$ beads, one with two $A$ and one $B$ bead,
	one with one $A$ and two $B$ beads, and one with three $B$ beads.
\end{example}

%--------------------------------------------------------------------------
\begin{example}
	We can use Polya's theorem to count chords in 12 tones that are equivalent under
	transposition by setting our set of colors to be $R = \{ r_0, r_1 \}$. We then disregard
	one of the colors, say $r_0$, by setting $w(r_0) = 1$ and $w(r_1) = C$. The group of
	transpositions is just $C_{12} = \mathbb{Z} / 12 \mathbb{Z}$, so we get
	\begin{equation}
		\begin{align}
			P_{C_{12}}(1 + x, \cdots, 1 + x^{12})
			& = \frac{1}{12} \sum_{d \in \{ 1, 2, 3, 4, 6, 12 \}} \varphi(d) x_d^{12 / d} \\
			& = \frac{1}{12} \left( x_1^{12} + x_2^{6} + 2 x_3^{4} + 2 x_4^{3} +
			2 x_6^{2} + 4 x_{12} \right) \\
			& = \frac{1}{12} \left[ (1 + C)^{12} + (1 + C^2)^{6} + 2 (1 + C^3)^{4} \right. \\
			& \qquad \qquad
			+ \left. 2 (1 + C^4)^{3} + 2 (1 + C^6)^{2} + 4 (1 + C^{12}) \right] \\
			& = C^{12} + C^{11} + \cdots \\
			& \qquad \qquad
			\cdots + 80 C^6 + 66 C^5 + 43 C^4 + 19 C^3 + 6 C^2 + C^1 + C^0 \enspace.
		\end{align}
	\end{equation}
	In particular, there are 43 tetrachords and 19 trichords that are transpositionally
	equivalent.
\end{example}

%--------------------------------------------------------------------------
\begin{example}
	\cite[53]{Reiner1985}
	More generally, the number of $k$-chords under the action of $C_n$ is
	\begin{equation}
		\frac{1}{n} \sum_{j | (n, k)} \varphi(j) {{n / j}\choose{k / j}} \enspace.
	\end{equation}
	Under the action of the dihedral group $D_{2n}$, we obtain
	\begin{equation}
		\begin{cases}
 			\frac{1}{2n} \left[ \sum_{j | (n, k)} \varphi(j) {{n / j}\choose{k / j}} +
 			n {{(n - 1) / 2} \choose{k / 2}} \right] & n \text{ odd} \\
 			\frac{1}{2n} \left[ \sum_{j | (n, k)} \varphi(j) {{n / j}\choose{k / j}} +
 			n {{n / 2} \choose{k / 2}} \right] & n \text{ even and } k \text{ even} \\
 			\frac{1}{2n} \left[ \sum_{j | (n, k)} \varphi(j) {{n / j}\choose{k / j}} +
 			n {{(n / 2) - 1} \choose{k / 2}} \right] & n \text{ even and } k \text{ odd}
 		\end{cases}
	\end{equation}
	In particular, the pattern inventory of $k$-chords in 12 tones that are equivalent under
	$\T_n \I$ is
	\begin{equation}
		P_{D_{24}}(1 + x, \cdots, 1 + x^{12}) = C^{12} + \cdots + 50 C^6 +
		38 C^5 + 29 C^4 + 12 C^3 + 6 C^2 + C^1 + C^0 \enspace.
	\end{equation}
\end{example}

%--------------------------------------------------------------------------
\begin{example}
	\cite[249]{Tucker1974}
	Consider the action of the dihedral group $D_8$ on the 16-element set of colorings in
	black and white of the corners of a square. The cycle indicator of $D_8$ is
	\begin{equation}
		P_{D_8}(x_1, x_2, x_3, x_4) =
		\frac{1}{8}(x_1^4 + 2x_1^2x_2 + 3x_2^2 + x_4^1) \enspace.
	\end{equation}
	Note that, in particular, we get no $x_3$ factors by Lagrange. The identity element
	in $D_8$ fixes all collorings, and is represented as a permutation by four cycles
	of length one. By Polya's formula, we make the substitution
	\begin{equation}
		x_1^4 = (B^1 + W^1)^4 = B^4 + 4 B^3 W + 6 B^2 W^2 + 4 B W^3 + W^4 \enspace,
	\end{equation}
	where the exponent outside the parenthesis is the number of cycles, and the exponents
	inside the parenthesis correspond to the lengths of the cycles, for each color.
	The element of order four in $D_8$ has one cycle of length four, so we get
	\begin{equation}
		x_4^1 = (B^4 + W^4)^1 = B^4 + W^4 \enspace,
	\end{equation}
	that is, the permutation $r = (1 \; 2 \; 3 \; 4)$ fixes the squares whose corners are
	all black or all white. Next, there are three elements in $D_8$ which comprise
	two cycles of length two, so each of those yield
	\begin{equation}
		x_2^2 = (B^2 + W^2)^2 = B^4 + 2 B^2 W^2 + W^4 \enspace.
	\end{equation}
	The last two elements in $D_8$ both have two cycles of length one, plus one cycle
	of length two, which gives
	\begin{equation}
		x_1^2x_2 = (B + W)^2 (B^2 + W^2) =
		B^4 + 2 B^3 W + 2 B^2 W^2 + 2 B W^2 + W^4 \enspace.
	\end{equation}
	Putting it all together, we get the following pattern inventory of orbits:
	\begin{equation}
		P_{D_8}(B + W, B^2 + W^2, B^3 + W^3, B^4 + W^4) =
		B^4 + 4 B^3 W + 6 B^2 W^2 + 4 B W^3 + W^4 \enspace.
	\end{equation}
\end{example}
