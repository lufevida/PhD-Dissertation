%--------------------------------------------------------------------------
\chapter{INTRODUCTION}
%--------------------------------------------------------------------------

%--------------------------------------------------------------------------
\begin{definition} \cite[99]{Rotman1967} Let $X$ be a set and $G$ a group. An \textbf{action} of $G$ on $X$ is a function $G \times X \to X$ given by $(g, x) \mapsto gx$, such that:
\begin{enumerate}[i.]
\item $(gh)x = g(hx)$ for all $g, h \in G$ and $x \in X$;
\item $1x = x$ for all $x \in X$, where $1 \in G$ is the identity.
\end{enumerate}
\end{definition}

%--------------------------------------------------------------------------
\begin{proposition} \cite[99]{Rotman1967} If a group $G$ acts on a set $X$ then, for every $g \in G$, the function $f_g : X \to X$ given by $f_g(x) = gx$ is a permutation of $X$. Further, the function $f : G \to S_X$ given by $f(g) = f_g$ is a homomorphism and, conversely, for any homomorphism $\phi : G \to S_X$, there is a corresponding group action given by $\phi(g)(x)$.
\end{proposition}

%--------------------------------------------------------------------------
\begin{theorem}[Cayley] \cite[96]{Rotman1967} Every group is isomorphic to a subgroup of the symmetric group $S_G$. In particular, if $|G| = n$, then $G$ is isomorphic to a subgroup of $S_n$.
\end{theorem}

%--------------------------------------------------------------------------
\begin{theorem} \cite[97]{Rotman1967} Let $H \leq G$ be a subgroup of finite index $n$. Then there exists a homomorphism $\phi : G \to S_n$ such that $\ker \phi \leq H$. In particular, when $H = \{ 1 \}$, we get Cayley's theorem.
\end{theorem}

%--------------------------------------------------------------------------
\begin{example} \cite[122]{DummitFoote2004} A group acts on itself by conjugation. Let $g, h \in G$ and $x \in G$. Then $1 x 1^{-1} = x$ and
$$
g(hx) = g(h x h^{-1}) = gh x h^{-1}g^{-1} = gh x gh^{-1} = (gh)x \enspace.
$$
It is also immediate from the above that a group acts on its power set by conjugation. In particular, a group acts on the set of all its subgroups.
\end{example}

%--------------------------------------------------------------------------
\begin{definition} \cite[100]{Rotman1967} If $G$ acts on $X$, then the \textbf{orbit} of $x \in X$ is the set
$$\mathcal{O}(x) = \{ gx : g \in G \} \subseteq X \enspace.
$$
We say an action is \textbf{transitive} if there is only one orbit. The \textbf{stabilizer} of $x$ in $G$ is the group
$$
G_x = \{ g \in g : gx = x \} \leq G \enspace.
$$
When a group acts on itself by conjugation, we call the orbits \textbf{conjugacy classes}. The stabilizer of some $g \in G$ is the \textbf{centralizer} of $g$ in $G$, denoted $C_G(g)$. When a group acts on the set of its subgroups by conjugation, the stabilizer of a subgroup $H \leq$ is the \textbf{normalizer} of $H$ in $G$, denoted by $N_G(H)$.
\end{definition}

%--------------------------------------------------------------------------
\begin{proposition} \cite[102]{Rotman1967} If $G$ acts on $X$, for $x_1, x_2 \in X$, the relation $x_1 \sim x_2$ given by $x_1 = gx_2$ is an equivalence relation. It follows immediately that the equivalence classes are the orbits of the action of $G$ on $X$ and that
$$
|X| = \sum_i |\mathcal{O}(x_i)| \enspace,
$$
where $x_i$ is a single representative from each orbit.
\end{proposition}

%--------------------------------------------------------------------------
\begin{theorem}[Orbit-Stabilizer] \cite[102]{Rotman1967} If $G$ acts on $X$, then for each $x \in X$
$$
|\mathcal{O}(x)| = [G : G_x] \enspace.
$$
\end{theorem}

%--------------------------------------------------------------------------
\begin{corollary} \cite[103]{Rotman1967} If $G$ is finite and acts on $X$, then the size of any orbit is a divisor of $|G|$.
\end{corollary}

%--------------------------------------------------------------------------
\begin{lemma}[Burnside] \cite[109]{Rotman1967} If $G$ acts on a finite set $X$, then the number of orbits $N$ is
$$
N = \frac{1}{|G|} \sum_{\tau \in G} \Fix(\tau) \enspace,
$$
where $\Fix(\tau)$ is the cardinality of the set of $x \in X$ that are fixed by $\tau$.
\end{lemma}

%--------------------------------------------------------------------------
\section{Polya's Enumeration Formula}
%--------------------------------------------------------------------------

The main application of Polya's enumeration formula in our context is as follows.

%--------------------------------------------------------------------------
\begin{example} \cite[??]{Tucker1974} Take an un-oriented cube and color its corners in black or white. Then
$$
b^8 + b^7w + 3b^6w^2 + 3b^5w^3 + 7b^4w^4 + 3b^3w^5 + 3b^2w^6 + bw^7 + w^8 \enspace,
$$
would represent the generating function, or pattern inventory of all distinct colorings of that cube, where the coefficient of $b^iw^j$ represents the particular number of colorings with $i$ black corners and $j$ white corners.
\end{example}
