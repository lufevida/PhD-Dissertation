%--------------------------------------------------------------------------
\chapter{INTRODUCTION}

%--------------------------------------------------------------------------
\vspace{3em}
\setlength{\epigraphwidth}{0.86\textwidth}
\setlength{\epigraphrule}{0pt}
\epigraph{
	\textit{Music has become an almost arbitrary matter, and composers will no longer be bound by laws and rules, but avoid the names of School and Law as they would Death itself...}
}{\vspace{2em}-- Johann Joseph Fux}
\vspace{1em}

%--------------------------------------------------------------------------
\section{Derivation and Polyphony}

%--------------------------------------------------------------------------
Derivation is the process of extracting ordered segments from rows in order to generate new compositional materials. It is a technique that dates back to the Second Viennese School. In its most incipient form, a composer may simply extract these segments from a row, and combine them to form another, as we see in Berg's \emph{Lulu}. The basic row used by Berg is $S = \{ 10, 2, 3, 0, 5, 7, 4, 6, 9, 8, 1, 11 \}$. In the Prologue, however, one is greeted with the row $\{ 10, 3, 4, 9, 2, 7, 8, 1, 0, 5, 6, 11 \}$, as depicted in Fig.~\ref{fig:berg}. It is clear that the segments that constitute the Prologue's row are ordered segments in the basic row form, and the fact that one row cannot obtained from another via row operations is irrelevant.

%--------------------------------------------------------------------------
\begin{figure}[htbp]
    \centering
%	\subfigure[Basic row]{
%		\begin{music}
%			\nostartrule
%			\startextract
%				\NOTes\nq{_bd_ecfge^fh^g^ji}\en
%				\setemptybar
%			\endextract
%		\end{music}
%	}
%	\subfigure[Derived row in the Prologue]{
%		\begin{music}
%			\nostartrule
%			\startextract
%				\NOTes\Qqbu{_b}{_e}{=e}{h}\Qqbu{d}{g}{^g}{^j}
%				\Qqbu{c}{f}{^f}{i}\en
%				\setemptybar
%			\endextract
%		\end{music}
%	}
	\caption[Berg's \emph{Lulu}]{Derived rows in Alban Berg's \emph{Lulu} \cite[182]{Starr1984}.}
	\label{fig:berg}
\end{figure}

%--------------------------------------------------------------------------
Less naive approaches to derivation ofted involve a derived row that will have more structure. In them, one will often see a combination matrix where derived and original rows are matched with some tewlve-tone or order operation of themselves, or both. This is illustrated in Ex.~\ref{ex:derivation}.

%--------------------------------------------------------------------------
\begin{example}
\label{ex:derivation}
We may use the basic row in \emph{Lulu} as motivation for a basic derivation procedure. The first step is to create a $2 \times 24$ array where the first row is $S$ followed by $\R \circ S$, and the second row is initially undefined.
\begin{equation}
    \left[
    \begin{array}{cccccccccccc|cccccccccccc}
        10 & 2 & 3 & 0 & 5 & 7 & 4 & 6 & 9 & 8 & 1 & 11 & 11 & 1 & 8 & 9 & 6 & 4 & 7 & 5 & 0 & 3 & 2 & 10 \\
        . & . & . & . & . & . & . & . & . & . & . & . & . & . & . & . & . & . & . & . & . & . & . & .
    \end{array}
    \right] \enspace.
\end{equation}
Next, one chooses an arbitrary segment, and separates it from the the top row by placing in in the bottom row:
\begin{equation}
    \left[
    \begin{array}{cccccccccccc|cccccccccccc}
        . & 2 & 3 & . & . & . & 4 & 6 & 9 & 8 & 1 & 11 & . & . & . & . & . & . & 7 & 5 & 0 & . & . & 10 \\
        10 & . & . & 0 & 5 & 7 & . & . & . & . & . & . & 11 & 1 & 8 & 9 & 6 & 4 & . & . & . & 3 & 2 & .
    \end{array}
    \right] \enspace.
\end{equation}
Let $T = \{ 10, 3, 4, 6, 9, 8, 1, 11, 7, 5, 0, 2 \}$. Then $T$ is a row derived from $S$. In particular, the ordered segment $\{ 10, 0, 5, 7 \}$ in $S$ is preserved by $\R \circ T$.
\end{example}

%--------------------------------------------------------------------------
It is of interest to note at this point that, in this kind of construction, the choice of a particular segment is already an important compositional decision. This choice bears relevance in that it extablishes motivic material, that is, the segment itself. It also potentially introduces complementary harmonic regions, one given by the segment, the other given by its set complement. Moreover, and perhaps more importantly, it presents an opportunity for exploring syntax. There are a multitude of ways in which a composer may obtain syntax from a simple derivation procedure such as the one given in the example above. One way would be to find an operation that makes the chosen segment invariant. In particular, it is easily checked that $S_1 = \{ 10, 0, 5, 7 \} = \R\T_5\I \circ S_1$. One can then extend Ex.~\ref{ex:derivation} into the combination array $[S \; | \; R \circ S \; | \; \R\T_5\I \circ S \; | \; T_5\I \circ S_1]$. In the extended array, the segment $S_1$ would be preserved, but the row derived from $\R\T_5\I \circ S$ would not be a transform of $T$. If the set complement of $S_1$ in $T$ were parsed to produce more than one harmonic region, then the complement of $S_1$ under this new derived row would produce different harmonic regions. This can be very pertinent compositionally, as one would be capable of producing contrasting harmonic regions while maintaining motivic coherence under the $S_1$ segment.

Yet another way of generating syntax from derivation would be to follow $S$ with $T$ itself. One would then derive a new row from $T$, say $Q$, and eventually follow $T$ with $Q$. Repeating this procedure \emph{ad libitum} could generate many contrasting harmonic regions. In particular, this tipe of derivation is seen in Donald Martino's \emph{Notturno} of 1974, a composition that won the Pulitzer Prize in the following year \cite[181]{Starr1984}. If, by compostional choice, the chain of derived rows picked always the same order numbers, then a potential for rhythmic and agogic coherence could also be explored.

In one of the seminal academic works in the field of 12-tone theory, \cite{Starr1984} utilizes a mostly set-theoretic framework to understand and categorize rows and procedures involved in producing derivation, polyphony, and self-derived combination matrices. The main objective is similar to ours, in it has a bias toward unveiling self derivation, which unfortunately still remains a somewhat obscure topic. The set-theoretic approach revolves around the idea of looking at collections from the standpoint of their order constraints: a totally constrained set with no precedence contradictions is a 12-tone row; a completely unconstrained set of 12 tones represents the free aggregate; a maximally constrained one is what the author calls the simmultaneous aggregate, that is, a 12-tone cluster. Sets that live in between can often be projected in the middle and background of a composition, fact that amounts to a Schenkerian-flavored view of the whole process.

Mathematically, the ideas in \cite{Starr1984} translate into considering the set $U$ of all ordered pairs of pitch classes. There are 12 choices for the first position, and 12 choices for the second position. As both choices are independent, this set has cardinality $12^2 = 144$. An element of $U$ is called an order constraint, and a subset $C$ of $U$ is called a pitch-class relation. The latter can be viewed as a $12 \times 12$ matrix where the entry $c_{ij}$ is equal to one whenever $\{ i, j \} \in C$, and zero otherwise. One can the apply biwise operations to these matrices in a very computationally efficient manner. For any pitch classes $x$ and $y$, we define a relation $x \sim y$ on the power set of $U$ by the set inclusion of the element $\{ x, y \}$. A subset $C$ will then be reflexive if, whenever an element of $C$ (which is a set) contains the pitch class $x$, then $\{ x, x \} \in C$. In words, reflexivity means that if a reflexive collection $C$ of notes contains an element $x$, then $x$ precedes (and follows) itself in $C$. The free aggregate is a minimal reflexive subset of $U$ that contains all 12 tones. The relation $\sim$ will be symmetric if $\{ x, y \} \in C$ implies $\{ y, x \} \in C$, and antisymmetric whenever $\{ x, y \} \in C$ implies $\{ y, x \} \notin C$, for $x \ne y \in \mathbb{Z}/ 12 \mathbb{Z}$. Similarly, transitivity is defined as $\{ x, y \} \in C$ and $\{ y, z \} \in C$, then $\{ x, z\} \in C$; and trichotomy is defined as either $\{ x, y \} \in C$ or $\{ y, x \} \in C$ for any $x \ne y \in \mathbb{Z}/ 12 \mathbb{Z}$. The relation $\sim$ is, of course, an order relation on the set of 12 tones by definition. A partial order is one that is reflexive, transitive, and antisymmetric, while a total order (a row), is a partial order that satisfies trichotomy.

Often, pitch-class relations will contain many redundancies due to transitivity. In order to express these relations as oriented graphs, one must first remove, or prune such redundancies. This process can be reversed and a pitch-class relation can be extended to the point of its transitive closure. It is also common for a pitch-class relation to be absent of any order constraint involving both $\{ x, y \}$, in which case we say $x$ and $y$ are incomparable. Such $x$ and $y$ are bound to be struck together, or else be \emph{linearized} by the injection of some constraint that will make them comparable, as long as the is still a partial order, that is, as long as it does not introduce a symmetry, for instance. The set of all total orderings that can be linearized out of some partial order is called its total order class. In a completely analogous manner, one can \emph{verticalize} a pitch-class relation by removing constraints, and again minding that the result is still transitive and symmetric. We say a partial order covers another whenever the former is a verticalization of the latter. A simple procedure to guarantee that a verticalization will remain a partial order is to take its union with the free aggregate, then subject this union to an extension operation, thus providing reflexivity in the first step, as well as transitivity in the second. We can say the following about covering and about unions and intersections of pitch-class relations:

%--------------------------------------------------------------------------
\begin{theorem}
    \cite[193]{Starr1984}
    \begin{enumerate}[i.]
        \item Covering is transitive;
        \item A pitch-class relation is covered by its extension;
        \item If a pitch-class relation covers another, then the extension of the former covers the extension of the latter.
    \end{enumerate}
\end{theorem}

%--------------------------------------------------------------------------
\begin{theorem}
    \cite[194]{Starr1984}
    Let $A$ and $B$ be partial orders and denote by $\Toc(A)$ and $\Toc(B)$ their respective total order classes. Then
    \begin{equation}
        \Toc(A) \cap \Toc(B) = \Toc ( \Ext( A \cup B ) ) \enspace,
    \end{equation}
    where $\Ext$ is the extension operator.
\end{theorem}

%--------------------------------------------------------------------------
\begin{theorem}
    \cite[194]{Starr1984}
    The intersection of two partial orders is again a partial order.
\end{theorem}






































\newpage
%--------------------------------------------------------------------------
\section{The Mallalieu Property}

%--------------------------------------------------------------------------
Consider the 12-tone series $S = \{ 0, 1, 4, 2, 9, 5, 11, 3, 8, 10, 7, 6 \}$. This series has the remarkable property that, if we include a dummy $13^\text{th}$ element, then taking every $n^\text{th}$ element of $S$ produces a transposition of it.

%--------------------------------------------------------------------------
\begin{example}
	We have $S^* = \{ 0, 1, 4, \dots, 7, 6, * \}$. Then taking every zeroth order number of $S^* \mod 13$ yields $S^*$ itself. Taking every first order number yields the series $\{ 1, 2, 5, \dots, 8, 7, * \}$ which, upon removing the dummy symbol, becomes $\T_1 \circ S$. Repeating this procedure every $n^\text{th}$ order number gives the sequence of transforms $\{ \T_i \}_{i \in S}$.
\end{example}

%--------------------------------------------------------------------------
This most peculiar property, commonly called the \emph{mallalieu} property, was first discovered by Pohlman Mallalieu \cite[285]{Lewin1966}. It is natural to ask at this point how many different 12-tone rows are there sharing this property. Unfortunately, there is only one such 12-tone row class under $\T_n\M\I$. We phrase below a little differently an argument given in \cite[17]{Morris1976}.

%--------------------------------------------------------------------------
\cite[278]{Lewin1966} provides a way of looking at mallalieu rows from the standpoint of replacing, for any 12-tone row, its order-number row $\{ 0, 1, \dots, 11 \}$ by the array of integers $\{ 1, 2, \dots, 12, 0 \}$ modulo 13. It is easy to see that such an array has the same structure as the array $S_*$ we constructed above if we substitute the asterisk by the number 12 and consider multiplication as the group operation. Obviously, this is just the isomorphism between the integers modulo 12 and the group of units modulo 13. One of the advantages of this approach is that we can dispense with the extra symbol altogether and just use the indices from 1 to $p - 1$. We shall, however, still refer to the row of order numbers as $S^*$, the context making it clear whether we are constructing it with an asterisk or not. The process of taking every $n^\text{th}$ element of a 12-tone row becomes then just the aforementioned multiplicative group operation on order numbers, that is, multiplying order numbers by $k \pmod{13}$ is the same as taking every $k^\text{th}$ element of a row.

%--------------------------------------------------------------------------
\begin{example}
	Put $S = \{ 0, 1, \dots, 11 \}$ and $S^* = \{ 1, 2, \dots, 12 \}$. Then
	\begin{equation}
		\M_3 \circ S^* = \{ 3, 6, \dots, 10 \} \enspace,
	\end{equation}
	which corresponds to the row $R = \{ 2, 5, \dots, 9 \}$. The row $R$ can be equivalently constructed by placing an asterisk as the $13^\text{th}$ order number of $S$ and taking every third element. The fact that $R$ and $S$ are not related by $\T_n\M\I$ reflects the fact that neither $S$ nor $R$ have the mallalieu property.
\end{example}

%--------------------------------------------------------------------------
It should be of interest to many composers whether other $n$-TET systems are capable of producing mallalieu rows, and if so, how many. Unfortunately, answering this question is not as straightforward as the above discussion, since we can no longer rely on the isomorphism that constitutes the proof of \ref{lewin-mallalieu}. We shall reformulate this question at the end of the present chapter, after having covered more of what has been already done.

%--------------------------------------------------------------------------
If, on one hand, we only get one $\T_n\M\I$ row class with the mallalieu property in 12 tones, we do get considerably more row classes when we relax the requirement that a row produce a transposition of itself when taking every $n^\text{th}$ of its elements. This idea is explored in part by \cite{Mead1989}, however without specifying any combinatorial aspect (in the mathematical sense) of this generalization. Moreover, we can certainly go beyond \cite{Mead1989} and investigate, in 12 tones, what an extension of the mallalieu property could yield under operations other than transposition.
