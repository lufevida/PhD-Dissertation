%--------------------------------------------------------------------------
\chapter{INTRODUCTION}

%--------------------------------------------------------------------------
In \ref{rahn-common-tone}, Rahn proves the second assertion as follows. We must count the occurrences of pairs of pitch classes that are interchanged by the operation at
hand and double them, for if $x$ maps onto $y$ under some $\T_n\I$, then certainly
$y$ maps onto $x$ under the same operation, given that every inversion operation
has order two. In addition to that, we must account for the occurrences of pitch
classes that may map onto themselves under the aforementioned operation. For any
pair $a \ne b \in S$, it follows $a$ and $b$ are exchanged by some operation
$\T_n\I$ whenever both $\T_n\I(a) = b$ and $\T_n\I(b) = a$ hold. Since
$\T_n\I(a) = -a + n$ and similarly $\T_n\I(b) = -b + n$, if the pair is exchanged,
we must have $-a + n = b$ and $-b + n = a$ both true. Adding the last two
expressions and yields $a + b = n$, which is the first set in the right-hand side
of the formula. As discussed above, the cardinality of this set must be doubled.
We have for any $a$ that $\T_n\I(a) = a + n$, hence $a = \T_n\I(a) \iff a = -a + n$,
that is, whenever $2a = n$. That is the second set in the formula. Finally,
for any pair $(a, n)$ such that $a = \T_n\I(a)$, we also have
$a + 6 = -(a + 6) + n \iff 2a = n$, so that by the above it follows
$a + 6 = \T_n\I(a + 6)$. Thus the set $\{a \in S : 2a = n\}$ has cardinality
at most 2, proving the last assertion.

%--------------------------------------------------------------------------
We can demonstrate \ref{rahn-example}, as well as the omitted proof of \ref{rahn-common-tone} under transposition in a much simpler way with a little bit of abstract algebra. By observing
the cycle decomposition the each operation at hand, if $n = 3$, then we have
\begin{equation}
	\T_3\I = (0 \; 3) (1 \; 2) (4 \; 11) (5 \; 10) (6 \; 9) (7 \; 8) \enspace.
\end{equation}
Hence, under $\T_3\I$, every pitch-class in $S = \{ 0, 1, 4, 5, 8, 9 \}$ gets sent to the
complement of $S$. If the operation is, for instance, $\T_9$, then since
\begin{equation}
	\T_9 = (0 \; 9 \; 6 \; 3) (1 \; 10 \; 7 \; 4) (2 \; 11 \; 8 \; 5) \enspace,
\end{equation}
we get straightforwardly that $S = \{ 0, 1, 4, 5, 8, 9 \}$ shares three common tones with
$\T_9 \circ S$, namely $0 \mapsto 9$, $4 \mapsto 1$, and $8 \mapsto 5$.

%--------------------------------------------------------------------------
Consider the 12-tone series $S = \{ 0, 1, 4, 2, 9, 5, 11, 3, 8, 10, 7, 6 \}$. This series
has the remarkable property that, if we include a dummy $13^\text{th}$ element, then
taking every $n^\text{th}$ element of $S$ produces a transposition of it.

%--------------------------------------------------------------------------
\begin{example}
	We have $S^* = \{ 0, 1, 4, \dots, 7, 6, * \}$. Then taking every zeroth order number of
	$S^* \mod 13$ yields $S^*$ itself. Taking every first order number yields the series
	$\{ 1, 2, 5, \dots, 8, 7, * \}$ which, upon removing the dummy symbol, becomes
	$\T_1 \circ S$. Repeating this procedure every $n^\text{th}$ order number gives the
	sequence of transforms $\{ \T_i \}_{i \in S}$.
\end{example}

%--------------------------------------------------------------------------
This most peculiar property, commonly called the \emph{mallalieu} property, was first discovered by Pohlman Mallalieu \cite[285]{Lewin1966}.
It is natural to ask at this point how many different 12-tone rows are there sharing
this property. Unfortunately, there is only one such 12-tone row class under $\T_n\M\I$.
We phrase below a little differently an argument given in \cite[17]{Morris1976}.
