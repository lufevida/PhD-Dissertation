%--------------------------------------------------------------------------
\chapter{INTRODUCTION}

%--------------------------------------------------------------------------
\section{Preliminary Results}

%--------------------------------------------------------------------------
\begin{definition} \cite[99]{Rotman1967} \cite[41]{DummitFoote2004} Let $X$ be a set and $G$ a group. An \textbf{action} of $G$ on $X$ is a function $G \times X \to X$ given by $(g, x) \mapsto gx$, such that:
\begin{enumerate}[i.]
\item $(gh)x = g(hx)$ for all $g, h \in G$ and $x \in X$;
\item $1x = x$ for all $x \in X$, where $1 \in G$ is the identity.
\end{enumerate}
\end{definition}

%--------------------------------------------------------------------------
\begin{proposition} \cite[99]{Rotman1967} \cite[42]{DummitFoote2004} If a group $G$ acts on a set $X$ then, for every $g \in G$, the function $f_g : X \to X$ given by $f_g(x) = gx$ is a permutation of $X$. Further, the function $f : G \to S_X$ given by $f(g) = f_g$ is a homomorphism and, conversely, for any homomorphism $\phi : G \to S_X$, there is a corresponding group action given by $\phi(g)(x)$.
\end{proposition}

%--------------------------------------------------------------------------
\begin{theorem}[Cayley] \cite[96]{Rotman1967} \cite[120]{DummitFoote2004} Every group is isomorphic to a subgroup of the symmetric group $S_G$. In particular, if $|G| = n$, then $G$ is isomorphic to a subgroup of $S_n$.
\end{theorem}

%--------------------------------------------------------------------------
\begin{theorem} \cite[97]{Rotman1967} Let $H \leq G$ be a subgroup of finite index $n$. Then there exists a homomorphism $\phi : G \to S_n$ such that $\ker \phi \leq H$. In particular, when $H = \{ 1 \}$, we get Cayley's theorem.
\end{theorem}

%--------------------------------------------------------------------------
\begin{example} \cite[122]{DummitFoote2004} A group acts on itself by conjugation. Let $g, h \in G$ and $x \in G$. Then $1 x 1^{-1} = x$ and
$$
g(hx) = g(h x h^{-1}) = gh x h^{-1}g^{-1} = (gh) x (gh)^{-1} = (gh)x \enspace.
$$
It is also immediate from the above that a group acts on its power set by conjugation. In particular, a group acts on the set of all its subgroups.
\end{example}

%--------------------------------------------------------------------------
\begin{definition} \cite[100]{Rotman1967} \cite[112]{DummitFoote2004} If $G$ acts on $X$, then the \textbf{orbit} of $x \in X$ is the set
$$\mathcal{O}(x) = \{ gx : g \in G \} \subseteq X \enspace.
$$
We say an action is \textbf{transitive} if there is only one orbit. The \textbf{kernel} of the action is the set
$$
\{ g \in g : gx = x, \forall x \in X \} \enspace.
$$
We say an action is \textbf{faithful} if the kernel is the identity. The \textbf{stabilizer} of $x$ in $G$ is the group
$$
G_x = \{ g \in g : gx = x \} \leq G \enspace.
$$
When a group acts on itself by conjugation, we call the orbits \textbf{conjugacy classes}. The stabilizer of some $g \in G$ is the \textbf{centralizer} of $g$ in $G$, denoted $C_G(g)$. When a group acts on the set of its subgroups by conjugation, the stabilizer of a subgroup $H \leq$ is the \textbf{normalizer} of $H$ in $G$, denoted by $N_G(H)$.
\end{definition}

%--------------------------------------------------------------------------
\begin{proposition} \cite[102]{Rotman1967} \cite[114]{DummitFoote2004} If $G$ acts on $X$, for $x_1, x_2 \in X$, the relation $x_1 \sim x_2$ given by $x_1 = gx_2$ is an equivalence relation. It follows immediately that the equivalence classes are the orbits of the action of $G$ on $X$ and that
$$
|X| = \sum_i |\mathcal{O}(x_i)| \enspace,
$$
where $x_i$ is a single representative from each orbit.
\end{proposition}

%--------------------------------------------------------------------------
\begin{theorem}[Orbit-Stabilizer] \cite[102]{Rotman1967} If $G$ acts on $X$, then for each $x \in X$
$$
|\mathcal{O}(x)| = [G : G_x] \enspace.
$$
\end{theorem}

%--------------------------------------------------------------------------
\begin{corollary} \cite[103]{Rotman1967} If $G$ is finite and acts on $X$, then the size of any orbit is a divisor of $|G|$.
\end{corollary}

%--------------------------------------------------------------------------
\begin{proposition} \cite[123]{DummitFoote2004} The number of conjugates of a subset $S$ in a group $G$ is $|G : N_G(S)|$,  the index of the normalizer of $S$. In particular, the number of conjugates of an element $s$ is $|G : C_G(s)|$, the index of the centralizer of $s$.
\end{proposition}

%--------------------------------------------------------------------------
\begin{proposition} \cite[125]{DummitFoote2004} Let $\sigma, \tau \in S_n$. If
$$
\sigma = (a_1 \; a_2 \; \cdots a_j) (b_1 \; b_2 \; \cdots a_k) \cdots \enspace,
$$
then
$$
\tau \sigma \tau^{-1} = (\tau(a_1) \; \tau(a_2) \; \cdots \tau(a_j)) (\tau(b_1) \; \tau(b_2) \; \cdots \tau(b_k)) \cdots \enspace.
$$
\end{proposition}

%--------------------------------------------------------------------------
\begin{example} \cite[127]{DummitFoote2004} Let $\sigma \in S_n$ be an $m$-cycle. The number of conjugates of $\sigma$ is
$$
\frac{|S_n|}{|C_{S_n}(\sigma)|} = \frac{n (n - 1) (n - m + 1)}{m} \enspace,
$$
so that $|C_{S_n}(\sigma)| = m (n - m)!$. Since $\sigma$ commutes with its powers, and also with any permutation in $S_n$ whose cycles are disjoint from it, and there are $(n - m)!$ of those, the number computed above is the full centralizer of $\sigma$.
\end{example}

%--------------------------------------------------------------------------
\begin{example} \cite[132]{DummitFoote2004} The size of each conjugacy class in $S_n$ is
$$
\frac{n!}{\prod_{r}r^{n_r}n_r!} \enspace,
$$
where, for each $r$-cycle, we divide by $r$ to account for the cyclical permutations of elements within a cycle. Further, if there are $n_r$ cycles of length $r$, we divide by $n_r!$ to account for the different orders in which those cycles may appear.
\end{example}

%--------------------------------------------------------------------------
\begin{proposition} \cite[126]{DummitFoote2004} Two elements in $S_n$ are conjugate if and only if they have the same cycle type. The number of conjugacy classes of $S_n$ is the number of partitions of $n$.
\end{proposition}

%--------------------------------------------------------------------------
\begin{definition} \cite[133]{DummitFoote2004} An isomorphism from a group $G$ to itself is called an \textbf{automorphism} of $G$. The group under composition of all automorphisms of $G$ is denoted by $\Aut(G)$.
\end{definition}

%--------------------------------------------------------------------------
\begin{proposition} \cite[133]{DummitFoote2004} If $H$ is a normal subgroup of $G$, then the action of $G$ by conjugation on $H$ is, for each $g \in G$, an automorphism of $H$. The kernel of the action is $C_G(H)$. In particular, $G / c_G(H)$ is a subgroup of $\Aut(H)$.
\end{proposition}

%--------------------------------------------------------------------------
\begin{corollary} \cite[134]{DummitFoote2004} For any subgroup $H < G$ and $g \in G$, $H \cong gHg^{-1}$. Moreover, $N_G(H) / C_G(H)$ (and also $G/Z(G)$ when $G = H$) is isomorphic to a subgroup of $\Aut(H)$.
\end{corollary}

%--------------------------------------------------------------------------
\begin{definition} \cite[135]{DummitFoote2004} A subgroup $H < G$ is called \textbf{characteristic} if every automorphism of $G$ maps $H$ to itself.
\end{definition}

%--------------------------------------------------------------------------
\begin{proposition} \cite[135]{DummitFoote2004} Characteristic subgroups are normal. Unique subgroups of a given order are characteristic. A characteristic subgroup of a normal subgroup is normal. In particular, every subgroup of a cyclic group is characteristic.
\end{proposition}

%--------------------------------------------------------------------------
\begin{proposition} \cite[135]{DummitFoote2004} If $G$ is cyclic of order $n$, then $\Aut(G) \cong (\mathbb{Z} / n \mathbb{Z})^\times$, and $|\Aut(G)| = \varphi(n)$, where $\varphi$ is Euler's totient function.
\end{proposition}

%--------------------------------------------------------------------------
\begin{corollary} \cite[136]{DummitFoote2004} Let $|G| = pq$, with $p \leq q$ primes. If $p \nmid q - 1$, then $G$ is abelian. If, further, $p < q$, then $G$ is cyclic.
\end{corollary}

%--------------------------------------------------------------------------
\begin{proposition} \cite[136]{DummitFoote2004} If $|G| = p^n$, with $p$ an odd prime, then
$$
|\Aut(G)| = p^{n - 1} (p - 1)
$$
and the latter group is also cyclic. If $|G| = 2^n$ is cyclic, with $n \geq 3$, then
$$
\Aut(G) \cong \mathbb{Z} / 2 \mathbb{Z} \times \mathbb{Z} / 2^{n - 2} \mathbb{Z}
$$
and the latter is not cyclic, but has a cyclic subgroup of index 2. If $V$ is the elementary abelian group of order $p^n$, then $pv = 0$ for all $v \in V$ and $V$ is an $n$-dimensional vector space over the field $\mathbb{F}_p = \mathbb{Z} / p \mathbb{Z}$. The automorphisms of $V$ are the nonsingular linear transformations from $V$ to itself:
$$
\Aut(V) \cong GL(V) \cong GL_n(\mathbb{F}_p) \enspace,
$$
and if $F$ is the finite field of order $q$, then
$$
|GL_n(F)| = (q^n - 1) (q^n - q) (q^n - q^2) \cdots (q^n - q^{n - 1}) \enspace.
$$
For all $n \ne 6$, we have $\Aut(S_n) \cong S_n$. Finally, $\Aut(D_8) \cong D_8$ and $\Aut(Q_8) \cong S_4$.
\end{proposition}

%--------------------------------------------------------------------------
\begin{example} \cite[137]{DummitFoote2004} The Klein 4-group is the elementary abelian group of order 4. It follows $\Aut(V_4) \cong GL_2(\mathbb{F}_2)$, and $|\Aut(V_4)| = 6$. Since the action of $\Aut(V_4)$ on $V_4$ permutes the latter's 3 nonidentity elements, by order considerations we have
$$
\Aut(V_4) \cong GL_2(\mathbb{F}_2) \cong S_3 \enspace.
$$
\end{example}

%--------------------------------------------------------------------------
\begin{proposition} \cite[314]{DummitFoote2004} A finite subgroup of the multiplicative group of a field is cyclic. In particular, if $F$ is a finite field, then $F^\times$ is cyclic.
\end{proposition}

%--------------------------------------------------------------------------
\begin{corollary} \cite[314]{DummitFoote2004} For $p$ a prime, $(\mathbb{Z} / p \mathbb{Z})^\times$ is cyclic.
\end{corollary}
