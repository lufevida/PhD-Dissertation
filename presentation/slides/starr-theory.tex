
%------------------------------------------------------------------------
\begin{frame}
	\frametitle{Starr (1984)}
	\begin{block}{Definition}
		A totally constrained set with no precedence contradictions is a twelve-tone row. A completely unconstrained set of twelve tones represents the \textbf{free aggregate}. A maximally constrained one is called the \textbf{simultaneous aggregate} (a twelve-tone cluster).
	\end{block}
	\begin{block}{Definition}
		Consider the set $U$ of all ordered pairs of pitch classes -- this set has cardinality $12^2 = 144$. An element of $U$ is called an \textbf{order constraint}, and a subset $C$ of $U$ is called a \textbf{pitch-class relation}.
	\end{block}
\end{frame}

%------------------------------------------------------------------------
\begin{frame}
	\frametitle{Starr (1984)}
	\begin{block}{Proposition}
		Define a relation $x \sim y$ on the power set of $U$ by the set inclusion of the element $\{ x, y \}$. Then $\sim$ is an order relation on the set of twelve tones.
	\end{block}
	\begin{block}{Definition}
		A \textbf{partial order} is one that is reflexive, transitive, and antisymmetric, while a \textbf{total order} (a row), is a partial order that satisfies trichotomy.
	\end{block}
\end{frame}

%------------------------------------------------------------------------
\begin{frame}
	\frametitle{Starr (1984)}
	\begin{block}{Example}
		The free aggregate is a minimal reflexive subset of $U$ that contains all twelve tones.
	\end{block}
	\begin{block}{Definition}
		\textbf{Pruning} is the operation that removes redundancies due to transitivity from a pitch-class relation. Pruning can be reversed by \textbf{extension} to the point of its transitive closure. We say $x$ and $y$ are \textbf{incomparable} in $C$ if the latter lacks any order constraint involving both $\{ x, y \}$. \textbf{Linearizing} means injecting some constraint, as long as there remains a partial order. The set of all rows that can be linearized from some partial order is called its \textbf{total order class}. One can \textbf{verticalize} a pitch-class relation by removing constraints, as long as there remains a partial order. We say a partial order \textbf{covers} another whenever the former is a verticalization of the latter.
	\end{block}
\end{frame}

%------------------------------------------------------------------------
\begin{frame}
	\frametitle{Starr (1984)}
	\begin{block}{Example}
		To guarantee that a verticalization will remain a partial order, we take its union with the free aggregate (reflexivity), then subject this union to an extension operation (transitivity).
	\end{block}
	\begin{block}{Theorem}
		\begin{itemize}
        	\item Covering is transitive.
        	\item A pitch-class relation is covered by its extension.
        	\item If a pitch-class relation covers another, then the extension of the former covers the extension of the latter.
    	\end{itemize}
	\end{block}
\end{frame}

%------------------------------------------------------------------------
\begin{frame}
	\frametitle{Starr (1984)}
	\begin{block}{Theorem}
		Let $A$ and $B$ be partial orders and denote by $\Toc(A)$ and $\Toc(B)$ their respective total order classes. Then $A \cap B$ is again a partial order and
    	\begin{equation*}
        	\Toc(A) \cap \Toc(B) = \Toc(\Ext(A \cup B)) \enspace,
    	\end{equation*}
    	where $\Ext$ is the extension operator. Moreover, if $A_i$ is a finite sequence of $n$ partial orders, then
    	\begin{equation*}
        	\bigcap_{i = 0}^{n} \Toc(A_i) = \Toc \left[ \Ext\left ( \bigcup_{i = 0}^{n} A_i \right) \right] \enspace.
    	\end{equation*}
	\end{block}
\end{frame}

%------------------------------------------------------------------------
\begin{frame}
	\frametitle{Starr (1984)}
	\begin{block}{Theorem}
		Let $C$ be a pitch-class relation and $\{ a, b \}$ an element of $U$ such that $\{ a, b \} \in C$.
    	\begin{itemize}
        	\item If $F$ is a pitch-class operation, then $\{ F(a), F(b) \} \in F(C)$ if and only if $\{ a, b \} \in C$. In particular, if $\R(C)$ is the retrograde of $C$, then $\{ a, b \} \in \R(C)$ if and only if $\{ b, a \} \in C$.
        	\item If $C$ is totally ordered, then $\R(C) = (S \setminus D) \cup F$, where $S$ is the simultaneous aggregate and $F$ is the free aggregate.
        	\item If $C_1$ covers $C_2$, then $F(C_1)$ covers $F(C_2)$.
        	\item If $C$ is $F\R$-invariant, then all cycles in $F$ have length two.
    	\end{itemize}
	\end{block}
\end{frame}
