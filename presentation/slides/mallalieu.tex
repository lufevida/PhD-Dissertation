
%------------------------------------------------------------------------
\begin{frame}
	\frametitle{The Mallalieu Property}
	\begin{block}{Example}
	Put $S^* = \{ 0, 1, 4, 2, 9, 5, 11, 3, 8, 10, 7, 6, * \}$. Then taking every zeroth order number of $S^* \mod 13$ yields trivially $S^*$ itself. Taking every first order number yields the row $\{ 1, 2, 5, 3, 10, 6, 0, 4, 9, 11, 8, 7, * \}$ which, upon removing the dummy symbol, becomes $\T_1(S)$. Repeating this procedure for every $n^\text{th}$ order number gives the sequence of $\T_i$-transforms of $S$ where the indices of transposition are totally-ordered, and correspond to the elements of $S$.
	\end{block}
\end{frame}

%------------------------------------------------------------------------
\begin{frame}
\frametitle{The Mallalieu Property}
	\begin{block}{Proposition}
		A 12-tone row has the mallalieu property if and only if it is related by $\T_n\M\I$ to the row $S = \{ 0, 1, 4, 2, 9, 5, 11, 3, 8, 10, 7, 6 \}$.
	\end{block}
	\begin{block}{Proof}
		One direction is just checking that every $\T_n\M\I$ transform of $S$ possesses the mallalieu property. Conversely, if a row $R$ in its untransposed prime form has the mallalieu property, then there is a transposition that takes its order numbers in zeroth rotation, that is, the set $\{ 0, 1, 2, 3, 4, 5, 6, 7, 8, 9, 10, 11 \}$ to its order numbers in, say, first rotation, id est, the set $\{ 1, 3, 5, 7, 9, 11, 0, 2, 4, 6, 8, 10 \}$. We can write this transposition as a permutation $0 \mapsto 1, 1 \mapsto 3, 2 \mapsto 5, \cdots, 11 \mapsto 10 $, or in cycle notation as $\hat{\T}_k = ( 0 \; 1 \; 3 \; 7 \; 2 \; 5 \; 11 \; 10 \; 8 \; 4 \; 9 \; 6 )$. Note that $\hat{\T}_k$ is an operation on order numbers.
	\end{block}
\end{frame}

%------------------------------------------------------------------------
\begin{frame}
\frametitle{The Mallalieu Property}
	\begin{block}{Proof}
	Since $\hat{\T}_k$ corresponds to a transposition, there are only four candidates for $T_k$, its pitch-class domain counterpart, namely $k \in \{ 1, 5, 7, 11 \}$, because these are the only indices for which a transposition of pitch classes in cycle notation is a 12-cycle. Moreover, we do not need to consider the cases where $k \in \{5, 7, 11\}$, as $\T_5 = \M \circ \T_1$, $\T_7 = \M \I \circ \T_1$, and $\T_{11} = \I \circ \T_1$. Hence, without loss, we can set $k = 1$. But then $S$ is the only row in untransposed prime form where $\T_1$ induces the permutation $\hat{\T}_k$ from its order numbers in zeroth rotation to its order numbers in first rotation. To see that, one needs to equate the cycles of $\hat{\T}_k$ with those of $\T_1$.
	\end{block}
\end{frame}

%------------------------------------------------------------------------
\begin{frame}
	\frametitle{The Mallalieu Property}
	\begin{block}{Example}
	Let $S = \{ 0, 1, 2, 3, 4, 5, 6, 7, 8, 9, 10, 11 \}$ and write $S^* = \{ 1, 2, 3, 4, 5, 6, 7, 8, 9, 10, 11, 12 \}$. Then $\M_3(S^*) = \{ 3, 6, 9, 12, 2, 5, 8, 11, 1, 4, 7, 10 \}$, which corresponds to the row $V = \{ 2, 5, 8, 11, 1, 4, 7, 10, 0, 3, 6, 9 \}$. The row $V$ can be constructed by placing an asterisk at the $13^\text{th}$ order number of $S$, then taking every third element. The fact that $V$ and $S$ are not related by $\T_n\M\I$ reflects the fact that neither $S$ nor $V$ have the mallalieu property.
	\end{block}
\end{frame}

%------------------------------------------------------------------------
\begin{frame}
	\frametitle{The Mallalieu Property}
	\begin{block}{Proposition}
	For $p$ a prime, every $(p - 1)$-TET system is capable of producing a mallalieu row.
	\end{block}
	\begin{block}{Example}
	The number of isomorphisms $\mathbb{Z} / 12 \mathbb{Z} \to (\mathbb{Z} / 13 \mathbb{Z})^\times$ is equal to
	\begin{equation*}
		|\Aut\big(\mathbb{Z} / 12 \mathbb{Z}\big)| = 4 \enspace.
	\end{equation*}
	If we map a generator of $\mathbb{Z} / 12 \mathbb{Z}$, say $\bar{1}$, to the generators of $(\mathbb{Z} / 13 \mathbb{Z})^\times$, namely $\bar{2}, \bar{6}, \bar{7}$ and $\overline{11}$, we obtain the four maps $i \pmod{12} \mapsto 2^i \pmod{13}$, $i \pmod{12} \mapsto 6^i \pmod{13}$, $i \pmod{12} \mapsto 7^i \pmod{13}$, and $i \pmod{12} \mapsto 11^i \pmod{13}$.
	\end{block}
\end{frame}

%------------------------------------------------------------------------
\begin{frame}
	\frametitle{The Mallalieu Property}
	\begin{block}{Example}
	Denote the first map by $\varphi$. Then
	\begin{equation*}
		\varphi(a + b) = 2^{a + b} = 2^a \cdot 2^b = \varphi(a) \cdot \varphi(b) \enspace,
	\end{equation*}
	so $\varphi$ is an isomorphism. Define $\varphi^{-1} : (\mathbb{Z} / 13 \mathbb{Z})^\times \to \mathbb{Z} / 12 \mathbb{Z}$ by
	$$\varphi^{-1}(\log i \pmod{13}) = i \pmod{12} \enspace.$$
	Let $S^* = \{ 1, 2, \dots, 11, 12 \}$ be a series of order numbers written multiplicatively. Then
	\begin{equation*}
		\varphi^{-1}(S^*) = \{ \log 1, \log 2, \dots, \log 12 \} \pmod{13} =
		\{ 0, 1, 4, \dots, 7, 6 \} \enspace.
	\end{equation*}
	\end{block}
\end{frame}
