
%------------------------------------------------------------------------
\begin{frame}
	\frametitle{Derivation from Aggregate Realizations}
	Let $S = \{ 0, 1, 7, 2 \} | \{ 10, 9 \} | \{ 11, 4, 8, 5 \} | \{ 3, 6 \}$ and consider the combination matrix below:
    \begin{equation*}
        A = \left[
        \begin{array}{c|c|c|c}
        	\{ 0, 1, 7, 2 \} & \{ 10, 9 \} & \{ 11, 4, 8, 5 \} & \{ 3, 6 \} \\
        	\{ 6, 3 \} & \{ 5, 8, 4, 11 \} & \{ 9, 10 \} & \{ 2, 7, 1, 0 \}
        \end{array}
        \right]
    \end{equation*}
	The first column of $A$ comprises the partial order $S_1 \cup \R(S_4)$, and this union maps onto its hexachordal complement under $\T_{11}\I$.
	\begin{equation*}
        \hat{A} = \left[
        \begin{array}{c|c|c|c}
        	\{ 0, 1, 7, 2 \} & \{ 10, 9 \} & \{ 11, 4, 8, 5 \} & \{ 3, 6 \} \\
        	\{ 6, 3 \} & \{ 5, 8, 4, 11 \} & \{ 9, 10 \} & \{ 2, 7, 1, 0 \} \\
        	\{ 11, 10, 4, 9 \} & \{ 1, 2 \} & \{ 0, 7, 3, 6 \} & \{ 8, 5 \} \\
        	\{ 5, 8 \} & \{ 6, 3, 7, 0 \} & \{ 2, 1 \} & \{ 9, 4, 10, 11 \}
        \end{array}
        \right]
    \end{equation*}
\end{frame}

%------------------------------------------------------------------------
\begin{frame}
	\frametitle{Derivation from Aggregate Realizations}
	Let $C_1, C_2, C_3, C_4$ be the four columns of the matrix $\hat{A}$. It follows the row $V = \{ 11, 0, 1, 6, 10, 4, 7, 2, 9, 5, 3, 8 \}$ has the property that $T \in \Toc \{ \Ext[ C_1 \cup \R\T_1\I(C_2) \cup \T_1\I(C_3) \cup \R(C_4) ] \}$. We can therefore derive $[V | \R\T_1\I(V) | \T_1\I(V) | \R(V)]$ from the columns of $\hat{A}$.
	\begin{equation*}
	\begin{adjustbox}{width=\textwidth}
        $\left[
        \begin{array}{cccccccccccc|cccccccccccc|c}
            & 0 & 1 &&&& 7 & 2 &&&& && 10 &&&&& 9 &&&&& & \\
            &&& 6 &&&&&&& 3 & & 5 && 8 & 4 & 11 &&&&&&& & \\
            \hline
            11 &&&& 10 & 4 &&& 9 &&& &&&&&&&&&&& 1 & 2 & \\
            &&&&&&&&& 5 && 8 &&&&&& 6 && 3 & 7 & 0 && & 2
        \end{array}
        \right. \cdots$
    \end{adjustbox}
    \end{equation*}
    \begin{equation*}
    \begin{adjustbox}{width=\textwidth}
        $\cdots \left.
        \begin{array}{c|cccccccccccc|cccccccccccc}
            & &&&&&&& 11 & 4 & 8 && 5 && 3 &&&&&&& 6 &&& \\
            & &&&&& 9 &&&&& 10 & &&&&& 2 & 7 &&&& 1 & 0 & \\
            \hline
            & && 0 & 7 & 3 && 6 &&&&& & 8 && 5 &&&&&&&&& \\
            2 & 2 & 1 &&&&&&&&&& &&&& 9 &&& 4 & 10 &&&& 11
        \end{array} \right]$
    \end{adjustbox}
    \end{equation*}
\end{frame}

%------------------------------------------------------------------------
\begin{frame}[fragile]
	\frametitle{Derivation from Aggregate Realizations}
	Let $S = \{ 0, 1, 7, 2 \} | \{ 10, 9, 11, 4 \} | \{ 8, 5, 3, 6 \} = S_1 | S_2 | S_3$ and consider the matrix $A = [S | \T_4(S) | \T_8(S)]^T$.
    \begin{equation*}
        A = \left[
        \begin{array}{cccc|cccc|cccc}
        	0 & 1 & 7 & 2 & 10 & 9 & 11 & 4 & 8 & 5 & 3 & 6 \\
        	4 & 5 & 11 & 6 & 2 & 1 & 3 & 8 & 0 & 9 & 7 & 10 \\
        	8 & 9 & 3 & 10 & 6 & 5 & 7 & 0 & 4 & 1 & 11 & 2
        \end{array}
        \right] \enspace.
    \end{equation*}
    Now let $V$ be in the total order class of the first columnar aggregate of $A$. Next, rewrite the first columnar aggregate of $A$ as the aggregate realization $A_1$.
    \begin{equation*}
    \begin{adjustbox}{width=\textwidth}
        $A_1 = \begin{tikzcd}
            & 0 \arrow[dr] && 1 \arrow[dr] && 7 \arrow[dr] && 2 \arrow[dr] & \\
            * \arrow[r] \arrow[dr] \arrow[ur] & 4 \arrow[r] & * \arrow[r] \arrow[dr] \arrow[ur] & 5 \arrow[r] & * \arrow[r] \arrow[dr] \arrow[ur] & 11 \arrow[r] & * \arrow[r] \arrow[dr] \arrow[ur] & 6 \arrow[r] & * \\
            & 8 \arrow[ur] && 9 \arrow[ur] && 3 \arrow[ur] && 10 \arrow[ur] &
        \end{tikzcd}$
    \end{adjustbox}
    \end{equation*}
\end{frame}

%------------------------------------------------------------------------
\begin{frame}
	\frametitle{Derivation from Aggregate Realizations}
	Any $V \in \Toc(A_1)$ must be a succession of augmented triads, say, $V = \{ 4, 8, 0, 9, 1, 5, 11, 7, 3, 2, 10, 6 \}$. We would like to linearize some transform of $V$ from the other columns of $A$. To verify that $A_2$ is a transform of $A_1$, check whether there is a base-four $\R\T_n\M\I$ operation that maps $S_1 \pmod 4$ onto $S_2 \pmod 4$.
    \begin{equation*}
        S_1 \pmod 4 = \{ 0, 1, 3, 2 \} = \T_2\I(\{ 2, 1, 3, 0 \}) = \T_2\I \circ S_2 \pmod 4 \enspace.
    \end{equation*}
    The elements in each of $A_1$'s columns are incomparable, so we pick from each column in any order we want. We can reduce all of $A$'s columns $\mod 4$ by construction, so it is enough to only consider each column's residue modulo four.
\end{frame}

%------------------------------------------------------------------------
\begin{frame}
	\frametitle{Derivation from Aggregate Realizations}
	Since $S_1 \pmod 4 = S_3 \pmod 4$, we can derive $V$ itself from $A_3$, and thus obtain the following derivation matrix, where the second column is $\T_2\I(V)$.
	    \begin{multline*}
        \left[
        \begin{array}{cccccccccccc|cccccc}
        	&& 0 && 1 &&& 7 && 2 && & 10 &&&&& 9 \\
        	4 &&&&& 5 & 11 &&&&& 6 & && 2 && 1 & \\
        	& 8 && 9 &&&&& 3 && 10 & & & 6 && 5 &&
        \end{array}
        \right. \cdots \\\\
        \cdots \left.
        \begin{array}{cccccc|cccccccccccc}
        	&& 11 && 4 & & & 8 &&&& 5 &&& 3 &&& 6 \\
        	3 &&&&& 8 & && 0 & 9 &&&& 7 &&& 10 & \\
        	& 7 && 0 && & 4 &&&& 1 && 11 &&& 2 &&
        \end{array}
        \right]
    \end{multline*}
\end{frame}

%------------------------------------------------------------------------
\begin{frame}[fragile]
	\frametitle{Derivation from Aggregate Realizations}
	Knowing that the columns of $A$ are related as aggregate realizations by the operation tuple $\mathcal{A} = [\T_0 \; \T_2\I \; \T_0]$, we can regard the $A_i$ as columnar aggregates, then take $\hat{A} = \Ext[\bigcup_i(\mathcal{A}_i \circ A_i)]$. Any row we are able to linearize from $\hat{A}$, will be in $\bigcap_i \Toc(A_i)$, and thus we can derive its $\mathcal{A}_i$-transform from each $i$-column of $A$.
	\begin{equation*}
        \hat{A} = \begin{tikzcd}
            & 0 \arrow[r] \arrow[ddr] & 1 \arrow[r] \arrow[dr] & 7 \arrow[r] \arrow[ddr] & 2 \arrow[dr] & \\
            * \arrow[r] \arrow[dr] \arrow[ur] & 4 \arrow[r] \arrow[ur] & 5 \arrow[r] \arrow[dr] & 11 \arrow[r] \arrow[ur] & 6 \arrow[r] & * \\
            & 8 \arrow[r] \arrow[ur] & 9 \arrow[r] \arrow[uur] & 3 \arrow[r] \arrow[ur] & 10 \arrow[ur] &
        \end{tikzcd}
    \end{equation*}
\end{frame}

%------------------------------------------------------------------------
\begin{frame}
	\frametitle{Derivation from Aggregate Realizations}
    It follows $\hat{V} = \{ 0, 1, 4, 8, 9, 5, 7, 2, 11, 3, 10, 6 \}$ can be linearized from $\hat{A}$.
    \begin{multline*}
        \left[
        \begin{array}{cccccccccccc|cccccc}
        	0 &&&& 1 & 7 &&& 2 &&&&&&& 10 && \\
        	&&& 4 &&& 5 & 11 &&& 6 && 2 &&&& 1 & \\
        	& 8 & 9 &&&&&&& 3 && 10 && 6 & 5 &&& 7
        \end{array}
        \right. \cdots \\\\
        \cdots \left.
        \begin{array}{cccccc|cccccccccccc}
        	9 &&& 11 && 4 && 8 &&&&& 5 &&& 3 & 6 & \\
        	& 3 &&& 8 && 0 && 9 &&& 7 &&&&&& 10 \\
        	&& 0 &&&&&&& 4 & 1 &&& 11 & 2 &&&
        \end{array}
        \right]
    \end{multline*}
\end{frame}
